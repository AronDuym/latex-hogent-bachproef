%%=============================================================================
%% Methodologie
%%=============================================================================

\chapter{\IfLanguageName{dutch}{Methodologie}{Methodology}}%
\label{ch:methodologie}

%% TODO: In dit hoofstuk geef je een korte toelichting over hoe je te werk bent
%% gegaan. Verdeel je onderzoek in grote fasen, en licht in elke fase toe wat
%% de doelstelling was, welke deliverables daar uit gekomen zijn, en welke
%% onderzoeksmethoden je daarbij toegepast hebt. Verantwoord waarom je
%% op deze manier te werk gegaan bent.
%% 
%% Voorbeelden van zulke fasen zijn: literatuurstudie, opstellen van een
%% requirements-analyse, opstellen long-list (bij vergelijkende studie),
%% selectie van geschikte tools (bij vergelijkende studie, "short-list"),
%% opzetten testopstelling/PoC, uitvoeren testen en verzamelen
%% van resultaten, analyse van resultaten, ...
%%
%% !!!!! LET OP !!!!!
%%
%% Het is uitdrukkelijk NIET de bedoeling dat je het grootste deel van de corpus
%% van je bachelorproef in dit hoofstuk verwerkt! Dit hoofdstuk is eerder een
%% kort overzicht van je plan van aanpak.
%%
%% Maak voor elke fase (behalve het literatuuronderzoek) een NIEUW HOOFDSTUK aan
%% en geef het een gepaste titel.

\section{Requirements-analyse}
\label{sec:RequirementsAnalyseBP}

De eerste stap in het effectieve onderzoek is het opstellen van een requirement-analyse, waarbij de eisen voor de integratie tools worden opgelijst en gerangschikt volgens de MoSCoW-methode. Voor het opstellen van een concrete en volledige lijst worden twee bronnen geraadpleegd. Ten eerste wordt voorgaande literatuurstudie gebruikt om bepaalde noodzakelijke eisen te identificeren. Daarna worden gesprekken gevoerd met de interne stakeholders van Axians, zodat eventuele aanvullingen, nieuwe eisen en exacte parameters worden toegevoegd. Tenslotte wordt er nog één dag voorzien voor het finaliseren van de requirements-analyse.

\section{Longlist}
\label{sec:LonglistBP}

Na het samenstellen van een requirementsanalyse is het belangrijk een longlist van integratie tools en eventuele maatwerkoplossingen samen te stellen op basis van deze
analyse en de informatie uit de literatuurstudie. Deze lijst wordt vervolgens gerangschikt aan de hand van de volledige lijst van requirements uit de requirements-analyse. Deze stap zal één dag in beslag nemen, afhankelijk van de grootte van de lijst requirements. De rangschikking wordt weergegeven in een tabel, waarbij de best scorende tools en maatwerkoplossingen zich bovenaan bevinden.

\section{Shortlist}
\label{sec:ShortlistBP}

Vervolgens, na het maken van een longlist, is de volgende stap in de methodologie het maken van een shortlist op basis van de longlist. Uit deze lijst zullen drie integratie tools of maatwerkoplossingen geselecteerd worden op basis van de vereisten uit de requirements-analyse. Deze drie tools zullen elk individueel verder onderzocht worden en uitgebreid besproken. Hierbij wordt er gekeken naar wat de functionaliteit is van elke tool, waarvoor de tool hoofdzakelijk wordt gebruikt, hoe de software in elkaar zit en het welke kosten er aan iedere tool verbonden zijn.

\section{Proof of concept}
\label{sec:Proof of conceptBP}

\subsection{Uitleg}
\label{sec:UitlegBP}

Voor de proof of concept van dit onderzoek zullen de integratie tools uit de shortlist getest worden om te zien welke tool het best voldoet aan de vooropgestelde eisen uit de requirements analyse. In de shortlist werd dit gedaan op een theoretische wijze, maar in de proof of concept worden deze tools praktisch getest door ze te gebruiken om een hypothetisch bedrijfsproces te optimaliseren. De exacte vereisten en welk proces zal worden getest met deze tools zal bepaald worden in samenspraak met Axians zodat het proof of concept toepasselijk is op de huidige situatie van het bedrijf.

\subsection{Manier van onderzoek}
\label{sec:Manier van onderzoekBP}

Bij de start van de proof of concept wordt er eerst samengezeten met de stakeholders van \\Axians om samen een praktische opdracht uit te werken die toepasselijk is om de integratie tools op een correcte manier te testen. Deze gesprekken en de feedbackloop voor het uitwerken van het plan van aanpak zal 7 dagen in beslag nemen. Na het opstellen van een praktische opdracht zullen de integratie tools individueel onderzocht worden om te testen hoe deze exact werken zodat er daarna vlot aan de proof of concept gewerkt kan worden. Tijdens deze praktische uitwerking wordt de exacte tijdsduur van de opdracht gelogd, terwijl iedere integratietool wordt geëvalueerd op basis van de criteria uit de requirements-analyse en persoonlijke ervaringen met iedere tool.

\section{Verwerking proof of concept data}
\label{sec:Verwerking proof of concept dataBP}

Na een theoretisch en praktisch onderzoek te doen naar de integratie tools worden de eindresultaten geëvalueerd. Hiervoor worden de eindproducten van iedere tool met elkaar vergeleken en krijgen ze elk een score op basis van vooropgestelde criteria uit de requirements-analyse en de persoonlijke feedback van de stakeholders van Axians. Indien de omzetting van een bedrijfsproces aan de hand van een integratie tool niet volledig is afgewerkt, wordt dit ook opgenomen in het rapport.

\section{Conclusie proof of concept}
\label{sec:Conclusie proof of conceptBP}

Na het verwerken van de data en het evalueren van de integratie tools wordt in de laatste fase van de proof of concept conclusies getrokken zodat de sterktes en gebreken van iedere tool duidelijk worden blootgelegd. Vervolgens worden deze tools gerangschikt op basis van de theoretische en praktische vereisten uit de requirements-analyse. Hierbij wordt ook rekening gehouden met onderzoeksvraag en alle subvragen. Deze conclusies samen met alle bevindingen worden uiteindelijk dan voorgelegd aan de stakeholders van Axians als een eindrapport.

