\chapter{\IfLanguageName{dutch}{Microsoft Power Automate}{Microsoft Power Automate}}
\label{ch:Microsoft}

De laatste integratie tool waarvan de resultaten zullen besproken worden is Microsoft Power Automate. Hieronder worden de requirements individueel besproken en krijgen ze een score tussen 1 (voldoet niet aan de verwachtingen) en 5 (voldoet aan alle verwachtingen) toegewezen. Op het einde van het hoofdstuk worden alle requirements en hun scores samengevat en weergegeven in een tabel met enkele extra berekeningen om het vergelijkingsproces te vereenvoudigen.

\section{Requirements}%
\label{RequirementsMicrosoft}

Om deze review iets vlotter leesbaar te maken zal Microsoft Power Automate afgekort worden tot Microsoft of MicrosoftPA gedurende deze review.

\subsection{Must have}%
\label{MustHaveMicrosoft}

\textbf{Veiligheid en betrouwbaarheid: De tool moet voldoen aan de relevante beveiligingsnormen (Welbekende ISO-normen, data-encryptie, Europese GDPR-wetgeving compliance).}

\vspace{\baselineskip}

Microsoft is wereldwijd bekend om hun veilige en betrouwbare producten en diensten. Hun data integratie tool Microsoft Power Automate draait in de cloud op het Microsoft Azure-platform. Hierdoor beschikt het data integratie platform over verschillende ISO-normen, sterke en flexibele data-encryptie en voldoet het platform ook aan de eisen van de Europese GDPR-wetgeving. Op vlak van ISO-normen beschikt MicrosoftPI over ISO/IEC 27001 voor informatiebeveiliging management, ISO/IEC 27018 voor de bescherming van persoonlijke gegevens in de cloud en ISO/IEC 27701 voor Privacy Information Management. Op vlak van data-encryptie gebruikt MicrosoftPA "in transit" (TLS 1.2+) en "at rest" (Azure Storage Service Encryption) om data te beveiligen. Microsoft biedt ook de mogelijkheid om zelf versleuteling in te stellen met persoonlijke encryptiesleutels (Customer Managed Keys). Voor autorisatie maakt Microsoft ook gebruik van Role-based access control (RBAC) voor een overzichtelijke en veilige toegangscontrole.


Voor dit requirement scoort Microsoft een 5.

\vspace{\baselineskip}

\textbf{Uitbreidbaarheid: De tool moet eenvoudig uitbreidbaar zijn met nieuwe modules en integraties naarmate de organisatie groeit.}

\vspace{\baselineskip}

MicrosoftPA beschikt over een ruim aanbod aan mogelijkheden op vlak van uitbreidingen. Zo kan het platform gebruik maken van verschillende Microsoft enterprise-tools om de kwaliteit en functionaliteit van de service te verbeteren. Voor deze services zijn vaak wel meerdere of hogere licenties nodig. MicrosoftPA geeft wel de mogelijkheid om eigen custom connectoren aan te maken. Hiermee kunnen de beschikbare data integraties kunnen worden uitgebreid voor bepaalde maatwerk data-integraties. In hoeverre de standaard licenties voor MicrosoftPA uitbreidbaarheid toelaten is onzeker. De limiet voor uitbreidbaarheid lijkt gekoppeld te zijn aan het plafond van de gekozen licentie, wat beperkingen kan opleggen op vlak van flexibiliteit. 


Voor dit requirement scoort Microsoft een 4.


\vspace{\baselineskip}

\textbf{Integratie van meerdere systemen: Het platform moet in staat zijn om verschillende applicaties en systemen met elkaar te verbinden (zoals ERP, CRM, HRM, legacy-systemen, databases) met elkaar te verbinden.}

\vspace{\baselineskip}

MicrosoftPA is in staat om verschillende applicaties en systemen met elkaar te kunnen verbinden aan de hand van data integraties. Voor 2 van de 3 data integraties is het zelf mogelijk om de data integraties te bouwen met enkel de ingebouwde componenten van Microsoft doordat de integratie plaatsvond tussen 2 microsoft services. Voor de andere data integratie moest gebruikgemaakt worden van een http component om verbinding te kunnen maken met Atlassian Tempo. Hiermee is Microsoft uitstekend voor alle data integraties die plaatsvinden tussen verschillende Microsoft systemen en kan het ook goed omgaan met systemen buiten het Microsoft ecosysteem. Het zou dan ook mogelijk moeten zijn om bijna iedere soort data-integratie te kunnen creëren.

\vspace{\baselineskip}

Bij de proof of concept van deze data integratie tool waren er wel enkele problemen naar boven gekomen. Door de beperking van de gebruikte licentie was het niet mogelijk om de data integratie uit te voeren en konden enkel hypothetische data integratie worden opgesteld doordat alle enterprise-level connectoren enkel uitvoerbaar zijn met premium licenties. De aanvraag voor een premium trial werd helaas ook geweigerd. Zie \ref{ch:licentie} voor meer informatie over de gebruikte licentie en zie bijlage: \ref{ch:Microsoft1}, \ref{ch:Microsoft2} en \ref{ch:Microsoft3} voor de hypothetische data integraties.

Voor dit requirement scoort Microsoft een 4.

\vspace{\baselineskip}

\textbf{Automatisering van processen: Manuele administratieve taken moeten geautomatiseerd kunnen worden om fouten te verminderen en bepaalde processen te versnellen.}

\vspace{\baselineskip}

MicrosoftPA beschikt over verschillende manieren voor de uitvoering van data integraties. Het beschikt over een Automated cloud flow voor de automatische uitvoering van data integraties aan de hand van een vooraf bepaalde trigger, een Instant cloud flow voor het manueel triggeren van een data integratie of een scheduled cloud flow waarbij de gebruiker zelf bepaalt wanneer en hoe een data integratie wordt uitgevoerd. Tenslotte beschikt het ook nog over een desktop flow voor het automatiseren van data integraties op de desktop omgeving in plaats van in de cloud. MicrosoftPA biedt ook een tool aan dat deze processen kan evalueren en optimaliseren indien de gebruiker dit wenst.


Voor dit requirement scoort Microsoft een 4.

\vspace{\baselineskip}
\textbf{Data-transformatie: Het platform moet data makkelijk kunnen converteren tussen verschillende formaten en structuren (bijv. XML ↔ JSON, CSV ↔ database records).}

\vspace{\baselineskip}

MicrosoftPA biedt een basisfunctionaliteit aan voor data-transformatie die goed is voor eenvoudige of simpele scenario’s met een lage complexiteit. Bij grootschalige of complexe data-transformaties kunnen er soms tekortkomingen zijn door een gebrek aan een dedicated conversie tool. Ook bij geneste data kunnen er complicaties tevoorschijn komen. Tenslotte biedt MicrosoftPA een slechte ondersteuning voor data afkomstig uit een CSV. Het platform beschikt niet over een native actie voor het omzetten van CSV data naar andere dataformaten. Hiervoor zal een eigen custom parsing script geschreven moeten worden.

Voor dit requirement scoort Microsoft een 3.

\vspace{\baselineskip}


\textbf{Monitoring en logging: Het systeem moet real-time monitoring en logboek functionaliteiten bieden om fouten en prestaties bij te houden.}

\vspace{\baselineskip}

MicrosoftPA biedt degelijke monitoring en logging aan bij individuele data integraties voor het evalueren van operationele data integraties en het monitoren van fouten. Er zijn echter wel beperkingen bij het monitoren en loggen van data integraties op grote schaal door een gebrek aan een centrale monitoring en logging omgeving. Voor grote of complexe omgevingen zijn aanvullende tools of integraties nodig om aan globale monitoring en logging te kunnen doen.

Voor dit requirement scoort Microsoft een 3.

\vspace{\baselineskip}

\subsection{Should have}%
\label{ShouldHaveMicrosoft}

\textbf{Kostenbeheersing: De tool moet betaalbaar zijn met een transparant kostenmodel (licenties, onderhoud, implementatie).}

\vspace{\baselineskip}

Op vlak van prijstransparantie scoort MicrosoftPA vrij goed. Het maakt gebruik van een abonnementsstructuur waarbij alle betalende opties een vaste zichtbare prijs tonen. Het jammere bij deze abonnementsstructuur is dat deze abonnementen enkel op jaarbasis te verkrijgen zijn, wat zorgt voor een enorme beperking in flexibiliteit.  De prijzen voor deze service lijken ook langs de hogere kant te liggen als er niet gekozen wordt voor het individuele abonnement. Het positieve hierbij is wel dat er geen verbruikskosten worden gerekend bovenop de basisprijs.

\vspace{\baselineskip}

De abonnementsstructuur bestaat uit 3 opties waaruit de klant kan kiezen, elk aangepast naar de mate van specifieke doelgroepen. Hieronder worden de 3 abonnement modellen kort toegelicht voor een idee te geven van hun schaal en wat ze exact aanbieden:

\begin{itemize}
    \item Power Automate Premium: \$15.00 per gebruiker per maand op jaarbasis. Gericht naar individuele personen.
    \item Power Automate Process: \$150.00 per bot per maand op jaarbasis. Gericht naar bedrijven die hun belangrijke bedrijfsprocessen willen automatiseren.
    \item Power Automate Hosted Process: \$215.00 per bot per maand op jaarbasis. Gericht naar bedrijven die hun belangrijke bedrijfsprocessen willen automatiseren aan de hand van een virtuele machine dat gehost wordt op het Microsoft Azure platform.
\end{itemize}

Tenslotte is er ook de mogelijkheid om MicrosoftPA gratis voor 30 tot 90 dagen uit te proberen om een inkijk te geven in het ecosysteem van MicrosoftPA. Na deze periode wordt het trial-account behouden, maar worden alle flows met premium features stopgezet en wordt het account omgezet naar een gratis versie.

Voor dit requirement scoort Microsoft een 4.

\vspace{\baselineskip}

\textbf{Schaalbaarheid: De tool moet eenvoudig schaalbaar zijn naargelang de wensen van de organisatie.}

\vspace{\baselineskip}

Het is onzeker hoe schaalbaar MicrosoftPA is buiten de vaste licentiemodellen. Volgens de verwoording van Microsoft zelf lijkt de schaalbaarheid vast te hangen aan het standaard licentiemodel en is er geen mogelijkheid om te schalen buiten de vernoemde licentiemodellen. Het is wel mogelijk om een bepaalde vorm van schaalbaarheid toe te passen door MicrosoftPA te gebruiken in samenwerking met andere Microsoft enterprise -tools. Tenslotte kunnen er ook bepaalde beperkingen zijn op bepaalde connectoren in de vorm van request-limieten en doorvoersnelheid. Dit kan mogelijks voor problemen zorgen bij grootschalige bedrijven die op een constante basis gebruikmaken van bepaalde connectoren. Dit alles zorgt voor een zeer beperkte flexibiliteit op vlak van schaalbaarheid indien de licentiemodellen niet ideaal zijn voor de schaal van het bedrijf.


Voor dit requirement scoort Microsoft een 3.

\vspace{\baselineskip}

\textbf{Connectoren en adapters: Het platform moet kant-en-klare connectoren bieden voor veelgebruikte systemen (bijv. SAP, Microsoft Dynamics, Oracle).}
\vspace{\baselineskip}

MicrosoftPA beschikt over een ruim aanbod aan kant-en-klare connectoren om con-
necties te maken met welgekende systemen zoals SAP, Amazon, Microsoft Dyna-
mics, Oracle, Salesforce, Azure en andere grote bedrijven. Desondanks beschikt ook Microsoft niet over een connector om een rechtstreekse verbinding te maken met Atlassian Tempo en zal hiervoor zelf een connector aangemaakt worden. Waar MicrosoftPA vooral in uitblinkt is bij de connectie binnen het eigen Microsoft-ecosysteem hierbij worden talloze connectoren voorzien voor de verschillende Microsoft systemen met elkaar te verbinden. Voor meer niche en minder populaire systemen waar Microsoft geen connector voor heeft, biedt het de mogelijkheid om zelf je eigen custom connectoren aan te maken voor het gebruik bij de data integraties.


Voor dit requirement scoort Microsoft een 4.

\vspace{\baselineskip}

\textbf{Prestaties: De tool moet performant zijn bij het uitvoeren van bepaalde veelgebruikte data integraties, maar niet alle data integratie processen.}

\vspace{\baselineskip}

Power Automate biedt goede prestaties voor veelvoorkomende en lichtgewicht integraties, zeker binnen het Microsoft-ecosysteem. Voor zwaardere of grootschalige data integraties kunnen de prestaties tegenvallen vanwege limieten of haperingen bij het verwerken van deze data. Deze limieten en haperingen kunnen op hun beurt ervoor zorgen dat data integraties vertragen of zelfs voor zorgen dat er fouten ontstaan in het verwerkingsproces.

Voor dit requirement scoort Microsoft een 3.

\vspace{\baselineskip}

\textbf{Onderhoudsvriendelijkheid: De tool moet eenvoudig te updaten en te onderhouden zijn met minimale downtime.}

\vspace{\baselineskip}

Microsoft biedt een service level agreement van 99.99 \% uptime aan, maar beschikt enkel over basisfunctionaliteiten op vlak van revisie geschiedenis en toont het enkel wie de laatste wijzigingen heeft aangebracht voor publicatie, maar beschikt het niet over een diepgaand of gedetailleerde revisiegeschiedenis of wijzigingslogs. Indien een bedrijf deze functionaliteiten toch wenst kan er wel gebruikgemaakt worden van integraties met Power Platform ALM of Microsoft 365 Compliance Center.


Voor dit requirement scoort Microsoft een 3.

\vspace{\baselineskip}

\subsection{Could have}%
\label{CouldHaveMicrosoft}

\textbf{Hybrid en multi-cloud ondersteuning: Het platform moet applicaties kunnen integreren die draaien in verschillende omgevingen (on-premise, private cloud, public cloud).}

\vspace{\baselineskip}

Microsoft Power Automate ondersteunt on-premise, hybride en multi-cloud omgevingen. Voor on-premise integratie biedt Microsoft ondersteuning voor on-premise systemen via de On-premises Data Gateway. Hiermee kunnen workflows data ophalen of verzenden naar interne systemen zoals lokale databases, ERP’s, of fileshares. Voor cloud omgevingen biedt Microsoft uitstekende verbindingen voor alle data-integraties binnen het Microsoft ecosysteem. Voor andere cloud omgevingen biedt MicrosoftPA integratiemogelijkheden met applicaties zolang ze via API’s of connectors verbonden kunnen worden. Helaas zijn deze integraties buiten het Microsoft systeem minder diepgaand en beperkter dan de Microsoft cloud omgevingen. 


Voor dit requirement scoort Microsoft een 4.

\vspace{\baselineskip}

\textbf{Low-code / no-code interface: Het platform beschikt over een low-code of no-code interface voor het aanmaken van data integraties.}

\vspace{\baselineskip}

MicrosoftPA biedt sterke low-code capaciteiten en een gebruiksvriendelijk interface die veel integraties mogelijk maakt zonder diepgaande technische kennis. Langs de no-code kant lijkt het requirement niet volledig haalbaar bij complexere scenario's en voor meer geadvanceerde data integraties is technische expertise tamelijk nodig.

Voor dit requirement scoort Microsoft een 4.

\newpage

\begin{landscape}



\subsection{Eindoverzicht}%
\label{EindoverzichtMicrosoft}

\begin{table}[H]
\centering
\resizebox{\linewidth}{!}{% resize the table to fit page width
\begin{tabular}{|ll|}
\hline
\multicolumn{1}{|l|}{\textbf{Requirements uit de requirementsanalyse}}                                                                                                                                                     & \textbf{Score tussen 1 en 5} \\ \hline
\textbf{Must have}                                                                                                                                                                                                         &                              \\ \hline
\multicolumn{1}{|l|}{Veiligheid en betrouwbaarheid: De tool moet voldoen aan de relevante beveiligingsnormen (Welbekende ISO-normen, data-encryptie, Europese GDPR-wetgeving compliance).}                                 & 5                            \\ \hline
\multicolumn{1}{|l|}{Uitbreidbaarheid: De tool moet eenvoudig uitbreidbaar zijn met nieuwe modules en integraties naarmate de organisatie groeit.}                                                                         & 4                            \\ \hline
\multicolumn{1}{|l|}{Integratie van meerdere systemen: Het platform moet in staat zijn om verschillende applicaties en systemen met elkaar te verbinden (zoals ERP, CRM, HRM, legacy-systemen, databases) met elkaar te verbinden.} & 3                            \\ \hline
\multicolumn{1}{|l|}{Automatisering van processen: Manuele administratieve taken moeten geautomatiseerd kunnen worden om fouten te verminderen en bepaalde processen te versnellen.}                                       & 4                            \\ \hline
\multicolumn{1}{|l|}{Data-transformatie: Het platform moet data makkelijk kunnen converteren tussen verschillende formaten en structuren (bijv. XML ↔ JSON, CSV ↔ database records).}                                      & 3                            \\ \hline
\multicolumn{1}{|l|}{Monitoring en logging: Het systeem moet real-time monitoring en logboek functionaliteiten bieden om fouten en prestaties bij te houden.}                                                              & 3                            \\ \hline
\textbf{Should have}                                                                                                                                                                                                       &                              \\ \hline
\multicolumn{1}{|l|}{Kostenbeheersing: De tool moet betaalbaar zijn met een transparant kostenmodel (licenties, onderhoud, implementatie).}                                                                                & 4                            \\ \hline
\multicolumn{1}{|l|}{Schaalbaarheid: De tool moet eenvoudig schaalbaar zijn naargelang de wensen van de organisatie.}                                                                                                      & 3                            \\ \hline
\multicolumn{1}{|l|}{Connectoren en adapters: Het platform moet kant-en-klare connectoren bieden voor veelgebruikte systemen (bijv. SAP, Microsoft Dynamics, Oracle).}                                                     & 4                            \\ \hline
\multicolumn{1}{|l|}{Prestaties: De tool moet performant zijn bij het uitvoeren van bepaalde veelgebruikte data integraties, maar niet alle data integratie processen.}                                                    & 3                            \\ \hline
\multicolumn{1}{|l|}{Onderhoudsvriendelijkheid: De tool moet eenvoudig te updaten en te onderhouden zijn met minimale downtime.}                                                                                           & 3                            \\ \hline
\textbf{Could have}                                                                                                                                                                                                        &                              \\ \hline
\multicolumn{1}{|l|}{Hybrid en multi-cloud ondersteuning: Het platform moet applicaties kunnen integreren die draaien in verschillende omgevingen (on-premise, private cloud, public cloud).}                              & 4                            \\ \hline
\multicolumn{1}{|l|}{Low-code / no-code interface: Het platform beschikt over een low-code of no-code interface voor het aanmaken van data integraties.}                                                                   & 4                            \\ \hline
\multicolumn{1}{|l|}{\textbf{Gemiddelde}}                                                                                                                                                                                  & 3,62                            \\ \hline
\multicolumn{1}{|l|}{\textbf{Mediaan}}                                                                                                                                                                                     & 4                            \\ \hline
\end{tabular}
}
\caption{Microsoft Power Automate: Beoordeling van requirements op een schaal van 1 tot 5}
\end{table}

\end{landscape}

