%%=============================================================================
%% Samenvatting
%%=============================================================================

% TODO: De "abstract" of samenvatting is een kernachtige (~ 1 blz. voor een
% thesis) synthese van het document.
%
% Een goede abstract biedt een kernachtig antwoord op volgende vragen:
%
% 1. Waarover gaat de bachelorproef?
% 2. Waarom heb je er over geschreven?
% 3. Hoe heb je het onderzoek uitgevoerd?
% 4. Wat waren de resultaten? Wat blijkt uit je onderzoek?
% 5. Wat betekenen je resultaten? Wat is de relevantie voor het werkveld?
%
% Daarom bestaat een abstract uit volgende componenten:
%
% - inleiding + kaderen thema
% - probleemstelling
% - (centrale) onderzoeksvraag
% - onderzoeksdoelstelling
% - methodologie
% - resultaten (beperk tot de belangrijkste, relevant voor de onderzoeksvraag)
% - conclusies, aanbevelingen, beperkingen
%
% LET OP! Een samenvatting is GEEN voorwoord!

%%---------- Nederlandse samenvatting -----------------------------------------
%
% TODO: Als je je bachelorproef in het Engels schrijft, moet je eerst een
% Nederlandse samenvatting invoegen. Haal daarvoor onderstaande code uit
% commentaar.
% Wie zijn bachelorproef in het Nederlands schrijft, kan dit negeren, de inhoud
% wordt niet in het document ingevoegd.

\IfLanguageName{english}{%
\selectlanguage{dutch}
\chapter*{Samenvatting}
\lipsum[1-4]
\selectlanguage{english}
}{}

%%---------- Samenvatting -----------------------------------------------------
% De samenvatting in de hoofdtaal van het document

\chapter*{\IfLanguageName{dutch}{Samenvatting}{Abstract}}

Axians, onderdeel van de Vinci Energies Group, is een IT-provider die begin 2025 de overstap zal maken naar een nieuw ERP-systeem. Deze transitie legt druk op bestaande administratieve processen, zowel handmatige als geautomatiseerde, die momenteel niet optimaal zijn afgestemd op de groei en de interne structuur van het bedrijf. Dit onderzoek richt zich op de vraag welke integratie tool Axians het beste kan ondersteunen bij het stroomlijnen en optimaliseren van interne processen in samenhang met het nieuwe ERP-systeem. Deze integratie tools worden vervolgens getest in een proof of concept waarbij de functionaliteiten en prestaties van iedere tool geëvalueerd worden op basis van een hypothetisch bedrijfsproces. De verwachte resultaten tonen een rangschikking van de integratie tools op basis van criteria zoals veiligheid, functionaliteit, en kosten, met een specifieke aanbeveling per categorie. Deze evaluatie zal Axians voorzien van een duidelijke keuze voor een integratie-oplossing die de interne processen optimaliseert en de administratieve belasting vermindert. De conclusie van dit onderzoek stelt Axians in staat om de best passende integratie tool te selecteren, waardoor de organisatie beter kan inspelen op interne groei en veranderingen, met als einddoel de administratieve efficiëntie te verhogen en fouten te verminderen.
