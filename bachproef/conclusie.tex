%%=============================================================================
%% Conclusie
%%=============================================================================

\chapter{Conclusie}%
\label{ch:conclusie}

% TODO: Trek een duidelijke conclusie, in de vorm van een antwoord op de
% onderzoeksvra(a)g(en). Wat was jouw bijdrage aan het onderzoeksdomein en
% hoe biedt dit meerwaarde aan het vakgebied/doelgroep? 
% Reflecteer kritisch over het resultaat. In Engelse teksten wordt deze sectie
% ``Discussion'' genoemd. Had je deze uitkomst verwacht? Zijn er zaken die nog
% niet duidelijk zijn?
% Heeft het onderzoek geleid tot nieuwe vragen die uitnodigen tot verder 
%onderzoek?

Dit onderzoek tracht een antwoord te geven op de onderzoeksvraag: “Welke integratie tool is het meest geschikt voor Axians om interne processen
te optimaliseren en te stroomlijnen tijdens de overstap naar het nieuwe ERP-
pakket?”. Om een antwoord te kunnen geven op deze onderzoeksvraag is een vergelijkende studie opgesteld die 3 integratie tools geselecteerd uit een longlist op een theoretische en een praktische manier test en vergelijkt. Voor het vergelijken van de integratie tools is een requirementsanalyse en een proof of concept opgesteld. Hierbij werd er rekening gehouden met de hoofd- en subvragen vanuit de inleiding.

\section{Antwoord op de onderzoeksvraag en alle subvragen}%
\label{Antwoord onderzoeksvraag en subvragen}

Op vlak van veiligheid en betrouwbaarheid presteren Boomi en Microsoft Power Automate zeer goed. Beide beschikken over veiligheidscertificaten, een sterke data-encryptie en zijn compliant met de Europese-GDPR wetgeving. Microsoft Power Automate blonk hierbij uit met zijn verschillende veiligheidscertificaten en encryptie die gekoppeld zijn aan het Azure platform. Zapier scoorde op dit vlak veel lager door een gebrek aan een officieel ISO-certificaat. Ook waren er bezorgdheden bij Zapier over de exacte veiligheid van confidentiële data die over het platform kan worden verstuurd.

\vspace{\baselineskip}

Op vlak van uitbreidbaarheid en schaalbaarheid presteert Boomi het beste voor middelgrote ondernemingen die nood hebben aan grootschalige en complexe data integraties, gevolgd door Microsoft PA en tenslotte Zapier. Boomi met zijn abonnementsstructuur biedt veel flexibiliteit op vlak van uitbreidbaarheid en schaalbaarheid met hun abonnementsstructuur die op maat kan worden bepaald. Boomi focust met hun data integratie platform op het aanbieden van enterprise-level data integraties. Microsoft Power Automate biedt eerder beperkte mogelijkheden op vlak van uitbreidbaarheid en schaalbaarheid. Zo is het onduidelijk hoe flexibel het platform is buiten de vooraf opgestelde abonnementsstructuren, want Microsoft Power Automate beschikt niet over een abonnement dat op maat kan opgesteld worden. Op vlak van data integratie binnen het Microsoft systeem biedt Microsoft Power Automate uitstekende prestaties, maar kunnen data integraties buiten het ecosysteem beperkingen oplopen. Zapier biedt met zijn abonnementsstructuur een goede flexibiliteit op vlak van uitbreidbaarheid en schaalbaarheid, maar bleek het platform onvoldoende geschikt om aan de eisen van een middelgrote onderneming te voldoen die Enterprise-level data integraties wil implementeren. 

\vspace{\baselineskip}

Op vlak van kosten lijkt Zapier de goedkoopste optie te zijn, gevolgd door Microsoft Power Automate en tenslotte Boomi. Ondanks dat Zapier de goedkoopste optie blijkt te zijn, is er een enorme beperking in functionaliteit die deze iets lagere prijs niet kan rechtvaardigen. Zo beschikt Zapier over zeer beperkte monitoring en logging, moet het converteren van data zo goed als zelf door de gebruiker gedaan worden en zijn de bestaande connectoren te simplistisch voor het aanmaken van Enterprise-level data integratie. Microsoft Power Automate was de middennoot op vlak van prijs met een vaste abonnementsprijs. Op vlak van functionaliteit biedt Microsoft Power Automate goede features voor het aanmaken van custom connectoren en beschikt het over tools die complexere data integraties kunnen aanmaken. Enkel op vlak van monitoring en logging kan het platform tegenvallen met zijn beperkte middelen voor diepgaande analyses. Boomi bleek uiteindelijk de duurste en vaagste optie te zijn op vlak van prijs, maar beschikt wel over de meeste functionaliteit op vlak van connecties, conversies en dat monitoring en logging. Boomi toont geen vaste prijzen voor zijn abonnementen en geeft enkel een prijs voor zijn pay-as-you-go model waarbij nog eens extra verbruikskosten worden aangerekend die niet duidelijk worden vermeld. Ondanks deze vaagheid in prijs beschikt Boomi over de beste middelen voor het aanmaken en beheren van enterprise-level data integraties.

\vspace{\baselineskip}

Met al deze observaties kunnen we concluderen dat op een algemeen niveau Boomi de beste optie is voor Axians om aan Enterprise-level data integratie te doen voor het optimaliseren en stroomlijnen van hun interne bedrijfsprocessen. Microsoft Power Automate kan ook nog dienen als een goede integratie tool, maar dan vooral voor het optimaliseren van bedrijfsprocessen tussen applicaties en systemen in het Microsoft ecosysteem. Helaas kan Zapier niet aangeraden worden als oplossing doordat de data integratie tool niet beschikt over de nodige eisen om Axians te helpen in het optimaliseren en stroomlijnen van bedrijfsprocessen.

\section{Toekomst van dit onderzoek}%
\label{Toekomst van dit onderzoek}

Voor de toekomst van dit onderzoek raad ik Axians aan om Boomi en Microsoft Power Automate verder te bestuderen om te zien hoe goed deze integratie tools kunnen voldoen aan de exacte wensen en vereisten van Axians. Zo kan het zeker interessant zijn om deze tools te testen op een grotere schaal en/of met complexere data integratie om te zien of deze integratie tools nuttig kunnen zijn voor Axians. Ook kan het interessant zijn om deze integratie tools verder te vergelijken met andere data integratie platformen die niet geselecteerd werden uit de longlist, zoals Mulesoft of APPSeCONNECT om te zien of deze integratie tools in aanmerking komen voor gebruikt te worden.

\newpage

\begin{landscape}

\section{Gecombineerd eindoverzicht van alle integratie tools}%
\label{EindoverzichtTools}
    
\begin{table}[H]
\centering
\resizebox{\linewidth}{!}{% resize the table to fit page width
\begin{tabular}{|llll|}
\hline
\multicolumn{1}{|l|}{\textbf{Requirements uit de requirementsanalyse met scores tussen 1 en 5}}                                                                                                                            & \multicolumn{1}{l|}{\textbf{Boomi}} & \multicolumn{1}{l|}{\textbf{Zapier}} & \textbf{MicrosoftPA} \\ \hline
\textbf{Must have}                                                                                                                                                                                                         &                                     &                                      &                       \\ \hline
\multicolumn{1}{|l|}{Veiligheid en betrouwbaarheid: De tool moet voldoen aan de relevante beveiligingsnormen (Welbekende ISO-normen, data-encryptie, Europese GDPR-wetgeving compliance).}                                 & \multicolumn{1}{l|}{4}              & \multicolumn{1}{l|}{2}               & 5                     \\ \hline
\multicolumn{1}{|l|}{Uitbreidbaarheid: De tool moet eenvoudig uitbreidbaar zijn met nieuwe modules en integraties naarmate de organisatie groeit.}                                                                         & \multicolumn{1}{l|}{4}              & \multicolumn{1}{l|}{3}               & 4                     \\ \hline
\multicolumn{1}{|l|}{Integratie van meerdere systemen: Het platform moet in staat zijn om verschillende applicaties en systemen met elkaar te verbinden (zoals ERP, CRM, HRM, legacy-systemen, databases) met elkaar te verbinden.} & \multicolumn{1}{l|}{4}              & \multicolumn{1}{l|}{2}               & 3                     \\ \hline
\multicolumn{1}{|l|}{Automatisering van processen: Manuele administratieve taken moeten geautomatiseerd kunnen worden om fouten te verminderen en bepaalde processen te versnellen.}                                       & \multicolumn{1}{l|}{4}              & \multicolumn{1}{l|}{3}               & 4                     \\ \hline
\multicolumn{1}{|l|}{Data-transformatie: Het platform moet data makkelijk kunnen converteren tussen verschillende formaten en structuren (bijv. XML ↔ JSON, CSV ↔ database records).}                                      & \multicolumn{1}{l|}{5}              & \multicolumn{1}{l|}{1}               & 3                     \\ \hline
\multicolumn{1}{|l|}{Monitoring en logging: Het systeem moet real-time monitoring en logboek functionaliteiten bieden om fouten en prestaties bij te houden.}                                                              & \multicolumn{1}{l|}{5}              & \multicolumn{1}{l|}{3}               & 3                     \\ \hline
\textbf{Should have}                                                                                                                                                                                                       &                                     &                                      &                       \\ \hline
\multicolumn{1}{|l|}{Kostenbeheersing: De tool moet betaalbaar zijn met een transparant kostenmodel (licenties, onderhoud, implementatie).}                                                                                & \multicolumn{1}{l|}{3}              & \multicolumn{1}{l|}{4}               & 4                     \\ \hline
\multicolumn{1}{|l|}{Schaalbaarheid: De tool moet eenvoudig schaalbaar zijn naargelang de wensen van de organisatie.}                                                                                                      & \multicolumn{1}{l|}{4}              & \multicolumn{1}{l|}{3}               & 3                     \\ \hline
\multicolumn{1}{|l|}{Connectoren en adapters: Het platform moet kant-en-klare connectoren bieden voor veelgebruikte systemen (bijv. SAP, Microsoft Dynamics, Oracle).}                                                     & \multicolumn{1}{l|}{4}              & \multicolumn{1}{l|}{2}               & 4                     \\ \hline
\multicolumn{1}{|l|}{Prestaties: De tool moet performant zijn bij het uitvoeren van bepaalde veelgebruikte data integraties, maar niet alle data integratie processen.}                                                    & \multicolumn{1}{l|}{4}              & \multicolumn{1}{l|}{3}               & 3                     \\ \hline
\multicolumn{1}{|l|}{Onderhoudsvriendelijkheid: De tool moet eenvoudig te updaten en te onderhouden zijn met minimale downtime.}                                                                                           & \multicolumn{1}{l|}{4}              & \multicolumn{1}{l|}{3}               & 3                     \\ \hline
\textbf{Could have}                                                                                                                                                                                                        &                                     &                                      &                       \\ \hline
\multicolumn{1}{|l|}{Hybrid en multi-cloud ondersteuning: Het platform moet applicaties kunnen integreren die draaien in verschillende omgevingen (on-premise, private cloud, public cloud).}                              & \multicolumn{1}{l|}{4}              & \multicolumn{1}{l|}{2}               & 4                     \\ \hline
\multicolumn{1}{|l|}{Low-code / no-code interface: Het platform beschikt over een low-code of no-code interface voor het aanmaken van data integraties.}                                                                   & \multicolumn{1}{l|}{4}              & \multicolumn{1}{l|}{3}               & 4                     \\ \hline
\multicolumn{1}{|l|}{\textbf{Gemiddelde}}                                                                                                                                                                                  & \multicolumn{1}{l|}{4,08}              & \multicolumn{1}{l|}{2,62}               & 3,62                     \\ \hline
\multicolumn{1}{|l|}{\textbf{Mediaan}}                                                                                                                                                                                     & \multicolumn{1}{l|}{4}              & \multicolumn{1}{l|}{3}               & 4                     \\ \hline
\end{tabular}
}
\caption{Eindoverzicht van alle integratie tools}
\end{table}

\end{landscape}