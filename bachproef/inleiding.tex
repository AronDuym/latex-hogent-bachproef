%%=============================================================================
%% Inleiding
%%=============================================================================

\chapter{\IfLanguageName{dutch}{Inleiding}{Introduction}}%
\label{ch:inleiding}

\section{\IfLanguageName{dutch}{Probleemstelling}{Problem Statement}}%
\label{sec:probleemstelling}

Axians maakt deel uit van de Vinci energies groep en is een IT provider met zowel software als hardware implementaties en stapt begin 2025 over naar een nieuw ERP pakket. Omwille van interne groei, overnames en organisatorische wijzigingen staan er een aantal manuele en niet-manuele administratieve processen onder druk. Dit zorgt voor fouten en onnodige stress bij medewerkers. Volgens Axians zijn niet alle bedrijfsprocessen en gebruikte tools momenteel goed op elkaar afgestemd. Hierdoor kwam er een verzoek van het bedrijf om onderzoek te doen naar verschillende integraties tools op de markt en deze te vergelijken met elkaar. Bij dit onderzoek moet er hoofdzakelijk nadruk gelegd worden op de integratiemogelijkheden, functionaliteit, bruikbaarheid, onderhoudsvriendelijkheid, veiligheid en recurrente kosten.

\section{\IfLanguageName{dutch}{Onderzoeksvraag}{Research question}}%
\label{sec:onderzoeksvraag}

Gezien integratie tools een belangrijk aspect vormen van het ERP-proces, is het interessant om te onderzoeken. Welke integratie tool het meest geschikt is om Axians te helpen bij het optimaliseren van hun bedrijfsprocessen. De focus ligt hierbij op het vinden van een integratie tool die voldoet aan alle eisen van een grote internationale onderneming. Hiervoor luidt volgende onderzoeksvraag:

\begin{itemize}
  \item Welke integratie tool is het meest geschikt voor Axians om interne processen te optimaliseren en te stroomlijnen tijdens de overstap naar het nieuwe ERP-pakket?
\end{itemize}

\subsection{Subvragen}
\label{sec:SubvragenBP}
\begin{itemize}
  \item Hoe goed zorgen de tools voor veiligheid en betrouwbaarheid bij de uitwisseling van data tussen systemen?
  \item Welke integratie tool biedt de meeste mogelijkheden voor uitbreidbaarheid?
  \item Wat zijn de (lange termijn) kosten van elke integratie tool?
  \item Welke integratie tool biedt Axians de meeste functionaliteit?
  \item Welke integratie tool biedt de meeste mogelijkheden voor schaalbaarheid?
\end{itemize}

\section{\IfLanguageName{dutch}{Onderzoeksdoelstelling}{Research objective}}%
\label{sec:onderzoeksdoelstelling}

Dit onderzoek oogt op het ondersteunen van Axians bij het vinden van de optimale integratie tool voor het optimaliseren van het ERP-proces en ook algemene bedrijfsprocessen te verbeteren. Dit zal op zijn beurt dan ook ervoor moeten zorgen dat het bedrijf beter zal kunnen omgaan met interne groei en de last ervan op de administratie.

\section{\IfLanguageName{dutch}{Opzet van deze bachelorproef}{Structure of this bachelor thesis}}%
\label{sec:opzet-bachelorproef}

% Het is gebruikelijk aan het einde van de inleiding een overzicht te
% geven van de opbouw van de rest van de tekst. Deze sectie bevat al een aanzet
% die je kan aanvullen/aanpassen in functie van je eigen tekst.

De rest van deze bachelorproef is als volgt opgebouwd:

\vspace{\baselineskip}

In Hoofdstuk~\ref{ch:stand-van-zaken} wordt een overzicht gegeven van de stand van zaken binnen het onderzoeksdomein, op basis van een literatuurstudie. Hierbij worden enkele definities uitgelegd om voldoende achtergrondinformatie te geven over de onderwerpen die in deze bachelorproef aan bod komen en context te bieden over de verschillende aspecten van integratie tools.

\vspace{\baselineskip}

In Hoofdstuk~\ref{ch:methodologie} wordt de methodologie toegelicht en worden de gebruikte onderzoekstechnieken besproken om een antwoord te kunnen formuleren op de onderzoeksvragen.

\vspace{\baselineskip}

% TODO: Vul hier aan voor je eigen hoofstukken, één of twee zinnen per hoofdstuk

In Hoofdstuk~\ref{ch:Boomi}, \ref{ch:Zapier} en \ref{ch:Microsoft} worden de resultaten van de proof of concept besproken en wordt iedere integratie tool individueel beoordeelt op basis van de requirementsanalyse. Ieder hoofdstuk behandelt een individuele integratie tool

\vspace{\baselineskip}

In Hoofdstuk~\ref{ch:conclusie}, tenslotte, wordt de conclusie gegeven en een antwoord geformuleerd op de onderzoeksvragen. Daarbij wordt ook een aanzet gegeven voor toekomstig onderzoek binnen dit domein.