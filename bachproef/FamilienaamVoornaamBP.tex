%===============================================================================
% LaTeX sjabloon voor de bachelorproef toegepaste informatica aan HOGENT
% Meer info op https://github.com/HoGentTIN/latex-hogent-report
%===============================================================================

\documentclass[dutch,dit,thesis]{hogentreport}

% TODO:
% - If necessary, replace the option `dit`' with your own department!
%   Valid entries are dbo, dbt, dgz, dit, dlo, dog, dsa, soa
% - If you write your thesis in English (remark: only possible after getting
%   explicit approval!), remove the option "dutch," or replace with "english".

\usepackage{lipsum} % For blind text, can be removed after adding actual content

%% Pictures to include in the text can be put in the graphics/ folder
\graphicspath{{../graphics/}}

%% For source code highlighting, requires pygments to be installed
%% Compile with the -shell-escape flag!
%% \usepackage[chapter]{minted}
%% If you compile with the make_thesis.{bat,sh} script, use the following
%% import instead:
\usepackage[chapter,outputdir=../output]{minted}
\usemintedstyle{solarized-light}

%% Formatting for minted environments.
\setminted{%
    autogobble,
    frame=lines,
    breaklines,
    linenos,
    tabsize=4
}

%% Ensure the list of listings is in the table of contents
\renewcommand\listoflistingscaption{%
    \IfLanguageName{dutch}{Lijst van codefragmenten}{List of listings}
}
\renewcommand\listingscaption{%
    \IfLanguageName{dutch}{Codefragment}{Listing}
}
\renewcommand*\listoflistings{%
    \addcontentsline{toc}{chapter}{\listoflistingscaption}%
    \listof{listing}{\listoflistingscaption}%
}

% Other packages not already included can be imported here

%%---------- Document metadata -------------------------------------------------
% TODO: Replace this with your own information
\author{Ernst Aarden}
\supervisor{Dhr. F. Van Houte}
\cosupervisor{Mevr. S. Beeckman}
\title[Optionele ondertitel]%
    {Titel van de bachelorproef}
\academicyear{\advance\year by -1 \the\year--\advance\year by 1 \the\year}
\examperiod{1}
\degreesought{\IfLanguageName{dutch}{Professionele bachelor in de toegepaste informatica}{Bachelor of applied computer science}}
\partialthesis{false} %% To display 'in partial fulfilment'
%\institution{Internshipcompany BVBA.}

%% Add global exceptions to the hyphenation here
\hyphenation{back-slash}

%% The bibliography (style and settings are  found in hogentthesis.cls)
\addbibresource{bachproef.bib}            %% Bibliography file
\addbibresource{../voorstel/voorstel.bib} %% Bibliography research proposal
\defbibheading{bibempty}{}

%% Prevent empty pages for right-handed chapter starts in twoside mode
\renewcommand{\cleardoublepage}{\clearpage}

\renewcommand{\arraystretch}{1.2}

%% Content starts here.
\begin{document}

%---------- Front matter -------------------------------------------------------

\frontmatter

\hypersetup{pageanchor=false} %% Disable page numbering references
%% Render a Dutch outer title page if the main language is English
\IfLanguageName{english}{%
    %% If necessary, information can be changed here
    \degreesought{Professionele Bachelor toegepaste informatica}%
    \begin{otherlanguage}{dutch}%
       \maketitle%
    \end{otherlanguage}%
}{}

%% Generates title page content
\maketitle
\hypersetup{pageanchor=true}

%%=============================================================================
%% Voorwoord
%%=============================================================================

\chapter*{\IfLanguageName{dutch}{Woord vooraf}{Preface}}%
\label{ch:voorwoord}

%% TODO:
%% Het voorwoord is het enige deel van de bachelorproef waar je vanuit je
%% eigen standpunt (``ik-vorm'') mag schrijven. Je kan hier bv. motiveren
%% waarom jij het onderwerp wil bespreken.
%% Vergeet ook niet te bedanken wie je geholpen/gesteund/... heeft

Voor u ligt mijn bachelorproef met als titel ‘Een onderzoek en evaluatie van integratie tools voor ERP optimalisatie en de stroomlijning van bedrijfsprocessen in een snelgroeiende IT-omgeving’. Deze bachelorproef is de laatste stap in het afsluiten van mijn opleiding ‘Bachelor Toegepaste Informatica’ aan de Hogeschool Gent.

\vspace{\baselineskip}

Het schrijven van deze bachelorproef bleek een zeer uitdagende maar ook zeer leerrijke opdracht te zijn. Ik heb tijdens het proces en het opstellen van dit document veel geleerd over de wereld van integratie tools in de wereld van IT2Business en over mezelf als student die dit domein moest onderzoeken.

\vspace{\baselineskip}

Als eerste wil ik Laurens Coppens bedanken voor deze bachelorproef mogelijk te maken. Het is dankzij zijn connectie dat ik in contact ben gekomen met mijn co-promotor en dit onderzoek tot stand is gekomen.

\vspace{\baselineskip}

Vervolgens wil ik mijn co-promotor Frederik Trenson bedanken voor de deskundige begeleiding, waardevolle steun en feedback gedurende het opstellen van deze bachelorproef alsook voor zijn hulp bij het uitwerken van de praktische proef.

\vspace{\baselineskip}

Daarnaast wil ik ook mijn promotor Martine Van Audenrode bedanken voor haar waardevolle feedback en begeleiding gedurende het schrijven van deze bachelorproef.

\vspace{\baselineskip}

Ten slotte wil ik ook mijn mede-studenten, lectoren en ouders bedanken voor hun steun en aanmoediging tijdens dit hele proces.

\vspace{\baselineskip}

Als laatste hoop ik dat deze bachelorproef Axians kan helpen in hun zoektocht naar de ideale oplossing om hun administratieve bedrijfsprocessen te kunnen verbeteren aan de hand van een integratie tool.

\vspace{\baselineskip}

Aron Duym

%%=============================================================================
%% Samenvatting
%%=============================================================================

% TODO: De "abstract" of samenvatting is een kernachtige (~ 1 blz. voor een
% thesis) synthese van het document.
%
% Een goede abstract biedt een kernachtig antwoord op volgende vragen:
%
% 1. Waarover gaat de bachelorproef?
% 2. Waarom heb je er over geschreven?
% 3. Hoe heb je het onderzoek uitgevoerd?
% 4. Wat waren de resultaten? Wat blijkt uit je onderzoek?
% 5. Wat betekenen je resultaten? Wat is de relevantie voor het werkveld?
%
% Daarom bestaat een abstract uit volgende componenten:
%
% - inleiding + kaderen thema
% - probleemstelling
% - (centrale) onderzoeksvraag
% - onderzoeksdoelstelling
% - methodologie
% - resultaten (beperk tot de belangrijkste, relevant voor de onderzoeksvraag)
% - conclusies, aanbevelingen, beperkingen
%
% LET OP! Een samenvatting is GEEN voorwoord!

%%---------- Nederlandse samenvatting -----------------------------------------
%
% TODO: Als je je bachelorproef in het Engels schrijft, moet je eerst een
% Nederlandse samenvatting invoegen. Haal daarvoor onderstaande code uit
% commentaar.
% Wie zijn bachelorproef in het Nederlands schrijft, kan dit negeren, de inhoud
% wordt niet in het document ingevoegd.

\IfLanguageName{english}{%
\selectlanguage{dutch}
\chapter*{Samenvatting}
\lipsum[1-4]
\selectlanguage{english}
}{}

%%---------- Samenvatting -----------------------------------------------------
% De samenvatting in de hoofdtaal van het document

\chapter*{\IfLanguageName{dutch}{Samenvatting}{Abstract}}

Axians, onderdeel van de Vinci Energies Group, is een IT-provider die begin 2025 de overstap zal maken naar een nieuw ERP-systeem. Deze transitie legt druk op bestaande administratieve processen, zowel handmatige als geautomatiseerde, die momenteel niet optimaal zijn afgestemd op de groei en de interne structuur van het bedrijf. Dit onderzoek richt zich op de vraag welke integratie tool Axians het beste kan ondersteunen bij het stroomlijnen en optimaliseren van interne processen in samenhang met het nieuwe ERP-systeem. Deze integratie tools worden vervolgens getest in een proof of concept waarbij de functionaliteiten en prestaties van iedere tool geëvalueerd worden op basis van een hypothetisch bedrijfsproces. De verwachte resultaten tonen een rangschikking van de integratie tools op basis van criteria zoals veiligheid, functionaliteit, en kosten, met een specifieke aanbeveling per categorie. Deze evaluatie zal Axians voorzien van een duidelijke keuze voor een integratie-oplossing die de interne processen optimaliseert en de administratieve belasting vermindert. De conclusie van dit onderzoek stelt Axians in staat om de best passende integratie tool te selecteren, waardoor de organisatie beter kan inspelen op interne groei en veranderingen, met als einddoel de administratieve efficiëntie te verhogen en fouten te verminderen.


%---------- Inhoud, lijst figuren, ... -----------------------------------------

\tableofcontents

% In a list of figures, the complete caption will be included. To prevent this,
% ALWAYS add a short description in the caption!
%
%  \caption[short description]{elaborate description}
%
% If you do, only the short description will be used in the list of figures

\listoffigures

% If you included tables and/or source code listings, uncomment the appropriate
% lines.
\listoftables
\listoflistings

% Als je een lijst van afkortingen of termen wil toevoegen, dan hoort die
% hier thuis. Gebruik bijvoorbeeld de ``glossaries'' package.
% https://www.overleaf.com/learn/latex/Glossaries

%---------- Kern ---------------------------------------------------------------

\mainmatter{}

% De eerste hoofdstukken van een bachelorproef zijn meestal een inleiding op
% het onderwerp, literatuurstudie en verantwoording methodologie.
% Aarzel niet om een meer beschrijvende titel aan deze hoofdstukken te geven of
% om bijvoorbeeld de inleiding en/of stand van zaken over meerdere hoofdstukken
% te verspreiden!

%%=============================================================================
%% Inleiding
%%=============================================================================

\chapter{\IfLanguageName{dutch}{Inleiding}{Introduction}}%
\label{ch:inleiding}

\section{\IfLanguageName{dutch}{Probleemstelling}{Problem Statement}}%
\label{sec:probleemstelling}

Axians maakt deel uit van de Vinci energies groep en is een IT provider met zowel software als hardware implementaties en stapt begin 2025 over naar een nieuw ERP pakket. Omwille van interne groei, overnames en organisatorische wijzigingen staan er een aantal manuele en niet-manuele administratieve processen onder druk. Dit zorgt voor fouten en onnodige stress bij medewerkers. Volgens Axians zijn niet alle bedrijfsprocessen en gebruikte tools momenteel goed op elkaar afgestemd. Hierdoor kwam er een verzoek van het bedrijf om onderzoek te doen naar verschillende integraties tools op de markt en deze te vergelijken met elkaar. Bij dit onderzoek moet er hoofdzakelijk nadruk gelegd worden op de integratiemogelijkheden, functionaliteit, bruikbaarheid, onderhoudsvriendelijkheid, veiligheid en recurrente kosten. Dit omvat ook een overweging van een maatwerkoplossing voor integraties die specifiek aansluit op de behoeften van Axians.

\section{\IfLanguageName{dutch}{Onderzoeksvraag}{Research question}}%
\label{sec:onderzoeksvraag}

Gezien integratie tools een belangrijk aspect vormen van het ERP-proces, is het interessant om te onderzoeken. Welke integratie tool het meest geschikt is om Axians te helpen bij het optimaliseren van hun bedrijfsprocessen. De focus ligt hierbij op het vinden van een integratie tool die voldoet aan alle eisen van een grote internationale onderneming. Hiervoor luidt volgende onderzoeksvraag:

\begin{itemize}
  \item Welke integratie tool of maatwerkoplossing is het meest geschikt voor Axians om interne processen te optimaliseren en te stroomlijnen tijdens de overstap naar het nieuwe ERP-pakket?
\end{itemize}

\subsection{Subvragen}
\label{sec:SubvragenBP}
\begin{itemize}
  \item Hoe goed zorgen de tools voor veiligheid en betrouwbaarheid bij de uitwisseling van data tussen systemen?
  \item Welke integratie tool biedt de meeste mogelijkheden voor uitbreidbaarheid?
  \item Wat zijn de lange termijn kosten van elke tool, vergeleken met een maatwerkoplossing?
  \item Welke integratie tool biedt Axians de meeste functionaliteit?
  \item Welke integratie tool biedt de meeste mogelijkheden voor schaalbaarheid?
\end{itemize}

\section{\IfLanguageName{dutch}{Onderzoeksdoelstelling}{Research objective}}%
\label{sec:onderzoeksdoelstelling}

Dit onderzoek oogt op het ondersteunen van Axians bij het vinden van de optimale integratie tool voor het optimaliseren van het ERP-proces en ook algemene bedrijfsprocessen te verbeteren. Dit zal op zijn beurt dan ook ervoor moeten zorgen dat het bedrijf beter zal kunnen omgaan met interne groei en de last ervan op de administratie.

\section{\IfLanguageName{dutch}{Opzet van deze bachelorproef}{Structure of this bachelor thesis}}%
\label{sec:opzet-bachelorproef}

% Het is gebruikelijk aan het einde van de inleiding een overzicht te
% geven van de opbouw van de rest van de tekst. Deze sectie bevat al een aanzet
% die je kan aanvullen/aanpassen in functie van je eigen tekst.

De rest van deze bachelorproef is als volgt opgebouwd:

\vspace{\baselineskip}

In Hoofdstuk~\ref{ch:stand-van-zaken} wordt een overzicht gegeven van de stand van zaken binnen het onderzoeksdomein, op basis van een literatuurstudie. Hierbij worden enkele definities uitgelegd om voldoende achtergrondinformatie te geven over de onderwerpen die in deze bachelorproef aan bod komen en context te bieden over de verschillende aspecten van integratie tools.

\vspace{\baselineskip}

In Hoofdstuk~\ref{ch:methodologie} wordt de methodologie toegelicht en worden de gebruikte onderzoekstechnieken besproken om een antwoord te kunnen formuleren op de onderzoeksvragen.

\vspace{\baselineskip}

% TODO: Vul hier aan voor je eigen hoofstukken, één of twee zinnen per hoofdstuk

In Hoofdstuk~\ref{ch:Boomi}, \ref{ch:Zapier} en \ref{ch:Microsoft} worden de resultaten van de proof of concept besproken en wordt iedere integratie tool individueel beoordeelt op basis van de requirementsanalyse. Ieder hoofdstuk behandelt een individuele integratie tool

\vspace{\baselineskip}

In Hoofdstuk~\ref{ch:conclusie}, tenslotte, wordt de conclusie gegeven en een antwoord geformuleerd op de onderzoeksvragen. Daarbij wordt ook een aanzet gegeven voor toekomstig onderzoek binnen dit domein.
\chapter{\IfLanguageName{dutch}{Stand van zaken}{State of the art}}%
\label{ch:stand-van-zaken}

% Tip: Begin elk hoofdstuk met een paragraaf inleiding die beschrijft hoe
% dit hoofdstuk past binnen het geheel van de bachelorproef. Geef in het
% bijzonder aan wat de link is met het vorige en volgende hoofdstuk.

% Pas na deze inleidende paragraaf komt de eerste sectiehoofding.

\section{Literatuurstudie}%
\label{sec:literatuurstudieBP}

In het eerste hoofdstuk van de literatuurstudie worden enkele belangrijke termen omtrent ERP en integratie tools gedefinieerd. Vervolgens wordt het belang van ERP in grote ondernemingen besproken samen met een korte geschiedenis van hoe bedrijven vroeger werkten voor het ontstaan van ERP en hoe andere systemen geëvolueerd zijn naar het ERP-systeem. Vervolgens worden de problemen en uitdagingen in verband met BPM aangekaart. Tenslotte wordt er ook kort besproken welke verbeteringen er nog kunnen worden toegepast op ERP om het proces nog te optimaliseren en te verbeteren aan de hand van nieuwe technologieën.

\subsection{Definities}
\label{sec:DefinitiesBP}

\textbf{ERP}: Een ERP, ofwel enterprise resource planning verwijst naar de software die bedrijven gebruiken voor het uitvoeren van hun dagdagelijkse administratieve bedrijfsactiviteiten, zoals in de boekhouding of bij de aankoop of verkoop van goederen. Wat ook vaak gekoppeld wordt aan ERP is enterprise performance management, software die helpt bij het plannen, budgetteren, voorspellen en rapporteren van de financiële resultaten van een organisatie. \autocite{Oracle2017}

\vspace{\baselineskip}

ERP-systemen zorgen ervoor dat een grote hoeveelheid aan bedrijfsprocessen worden samengebracht en gesynchroniseerd met elkaar. Door de gedeelde transactiegegevens van een organisatie uit meerdere bronnen te verzamelen, elimineren ERP-systemen dubbele gegevens en bieden ze gegevensintegriteit met één enkele bron van waarheid. \autocite{Oracle2017}

\vspace{\baselineskip}

Volgens \textcite{Oracle2017}: zijn ERP-systemen essentieel bij grote of internationale bedrijven voor de correcte verwerking van alle administratieve taken binnen deze ondernemingen. \textcite{Oracle2017} beweert zelf dat ERP even onmisbaar is als elektriciteit voor bedrijven van deze schaal.

\vspace{\baselineskip}

\textbf{Integratie tool}: Een integratie tool is een instrument of software dat gebruikt wordt voor het combineren van gegevens uit verschillende ongelijksoortige bronnen om gebruikers een overzichtelijk beeld te geven van deze informatie. Integratie tools brengen kleinere componenten in één systeem samen zodat deze als één geheel kunnen samenwerken. \autocite{Microsoft2024}

\vspace{\baselineskip}

Volgens \textcite{Microsoft2024} helpen integratie tools bij het consolideren van alle soorten gegevens, gezien de groei, het volume en de verschillende formaten. \textcite{Microsoft2024} beweert ook door deze te combineren om te werken met één set gegevens, dat bedrijven interne afdelingen kunnen helpen om oog in oog te staan met strategieën en zakelijke beslissingen en bruikbare en overtuigende zakelijke inzichten te produceren voor succes op korte en lange termijn.

\vspace{\baselineskip}

\textbf{BPMS}: BPMS, ofwel Business Process Management Systems zijn softwareplatformen die de definitie, uitvoering en opvolging van bedrijfsprocessen ondersteunen. Een BPMS maakt het mogelijk om informatie te observeren en te loggen over de bedrijfsprocessen binnen het bedrijf. \autocite{grigori2004business}

\vspace{\baselineskip}

Volgens \textcite{grigori2004business} kan een goede analyse van de logs uit een BPMS belangrijke informatie opleveren en bedrijven helpen om de kwaliteit van hun bedrijfsprocessen en diensten te verbeteren.




\subsection{Het belang van ERP-systemen in een ondernemening}
\label{sec:Het belang van ERP-systemen in een ondernemeningBP}

ERP-systemen combineren het concept van bedrijfsproces integratie met een technisch platform dat bestaat uit een geïntegreerde database en modules voor verschillende functionele domeinen. ERP-systemen hebben hun oorsprong in de vroege jaren van de informatica in de jaren 1940 en zijn geëvolueerd via geïntegreerde controletools (1960s) en Material Requirements Planning (MRP)-systemen (1970s en 1980s). In de jaren 1990 tot 2000 kenden ERP-systemen aanvankelijk een monolithische architectuur, die vanaf de 2010s plaatsmaakte voor postmoderne ERP-systemen met meerdere platformen. \autocite{katuu2020enterprise}

\vspace{\baselineskip}

Deze evolutie weerspiegelt de aanpassingen van ERP-systemen aan interne en externe uitdagingen binnen ondernemingen, zoals stijgende verwachtingen van stakeholders en klanten, terwijl beschikbare middelen afnemen. Voor effectieve integratie en waardecreatie moeten ERP-systemen ingebed worden in een technologisch ecosysteem dat rekening houdt met institutionele strategieën en operaties. Hierbij is het essentieel om over te stappen van traditionele monolithische systemen naar cloudgebaseerde en postmoderne ERP-platformen die compatibel zijn met innovaties zoals kunstmatige intelligentie en Robotic Process Automation. \autocite{katuu2020enterprise}

\vspace{\baselineskip}

Deze inzichten onderstrepen de noodzaak voor organisaties om hun ERP-systemen voortdurend te innoveren en af te stemmen op nieuwe technologische mogelijkheden.

\vspace{\baselineskip}

ERP speelt ook een cruciale rol bij het integreren van informatie en processen binnen en tussen de verschillende afdelingen van een onderneming. Dit is met name waardevol voor grote organisaties met complexe structuren en uiteenlopende operaties. Oorspronkelijk waren ERP-systemen gericht op het ondersteunen van interne operationele processen, maar hun functionaliteit is aanzienlijk uitgebreid. Tegenwoordig functioneren ze als platforms die de gehele bedrijfsvoering kunnen verbinden en integreren met andere bedrijfstoepassingen, zoals Supply Chain Management (SCM) en Customer Relationship Management (CRM). \autocite{sheik2020enterprise}

\vspace{\baselineskip}

De brede toepasbaarheid van ERP-systemen heeft hun implementatie over diverse industrieën gestimuleerd en geleid tot toenemende aandacht van management experts en onderzoekers. Hoewel er al veel vooruitgang is geboekt, biedt het onderwerp nog steeds talrijke mogelijkheden voor verder onderzoek vanuit verschillende invalshoeken. Verdere studies kunnen bijdragen aan een beter begrip en innovatieve toepassingen van ERP binnen uiteenlopende organisatorische contexten. \autocite{sheik2020enterprise}

\vspace{\baselineskip}


\subsection{Het nut van data integratie binnen een ondernemening}
\label{Het nut van data integratie binnen een ondernemening}

%TODO Antwoord geven op de onderzoeksvraag en subvragen

\vspace{\baselineskip}

\subsection{Monolithische ERP vs Best of breed ERP}
\label{Monolithische ERP vs Best of breed ERP}

%TODO

\vspace{\baselineskip}


\subsection{Mogelijke verbeteringen in het ERP-proces}
\label{sec:Mogelijke verbeteringen in het ERP-procesBP}

Het bewustzijn van organisaties over veranderingen in ERP-systemen speelt een cruciale rol in het verhogen van klanttevredenheid. Slimme apparaten die realtime gegevens verschaffen over producten, kwaliteit en transport hebben niet alleen een aanzienlijke impact op de klantenservice, maar verbeteren ook het algemene organisatie management. Vooral de integratie van cloud-ERP met IoT biedt veelbelovende mogelijkheden voor zowel efficiënter management als verbeterde klantgerichte dienstverlening. \autocite{tavana2020iot}

\vspace{\baselineskip}

\section{Enterprise Data Integration platformen}%
\label{sec:Enterprise Data Integration platformen}

\subsection{Boomi}
\label{Boomi}

\subsection{Zapier}
\label{Zapier}

\subsection{Microsoft Power Automate}
\label{Microsoft Power Automate}


%%=============================================================================
%% Methodologie
%%=============================================================================

\chapter{\IfLanguageName{dutch}{Methodologie}{Methodology}}%
\label{ch:methodologie}

%% TODO: In dit hoofstuk geef je een korte toelichting over hoe je te werk bent
%% gegaan. Verdeel je onderzoek in grote fasen, en licht in elke fase toe wat
%% de doelstelling was, welke deliverables daar uit gekomen zijn, en welke
%% onderzoeksmethoden je daarbij toegepast hebt. Verantwoord waarom je
%% op deze manier te werk gegaan bent.
%% 
%% Voorbeelden van zulke fasen zijn: literatuurstudie, opstellen van een
%% requirements-analyse, opstellen long-list (bij vergelijkende studie),
%% selectie van geschikte tools (bij vergelijkende studie, "short-list"),
%% opzetten testopstelling/PoC, uitvoeren testen en verzamelen
%% van resultaten, analyse van resultaten, ...
%%
%% !!!!! LET OP !!!!!
%%
%% Het is uitdrukkelijk NIET de bedoeling dat je het grootste deel van de corpus
%% van je bachelorproef in dit hoofstuk verwerkt! Dit hoofdstuk is eerder een
%% kort overzicht van je plan van aanpak.
%%
%% Maak voor elke fase (behalve het literatuuronderzoek) een NIEUW HOOFDSTUK aan
%% en geef het een gepaste titel.

\section{Requirements-analyse}
\label{sec:RequirementsAnalyseBP}

De eerste stap in het effectieve onderzoek is het opstellen van een requirement-analyse, waarbij de eisen voor de integratie tools worden opgelijst en gerangschikt volgens de MoSCoW-methode. Voor het opstellen van een concrete en volledige lijst worden twee bronnen geraadpleegd. Ten eerste wordt voorgaande literatuurstudie gebruikt om bepaalde noodzakelijke eisen te identificeren. Daarna worden gesprekken gevoerd met de interne stakeholders van Axians, zodat eventuele aanvullingen, nieuwe eisen en exacte parameters worden toegevoegd. Tenslotte wordt er nog één dag voorzien voor het finaliseren van de requirements-analyse.

\section{Longlist}
\label{sec:LonglistBP}

Na het samenstellen van een requirementsanalyse is het belangrijk een longlist van integratie tools en eventuele maatwerkoplossingen samen te stellen op basis van deze
analyse en de informatie uit de literatuurstudie. Deze lijst wordt vervolgens gerangschikt aan de hand van de volledige lijst van requirements uit de requirements-analyse. Deze stap zal één dag in beslag nemen, afhankelijk van de grootte van de lijst requirements. De rangschikking wordt weergegeven in een tabel, waarbij de best scorende tools en maatwerkoplossingen zich bovenaan bevinden.

\section{Shortlist}
\label{sec:ShortlistBP}

Vervolgens, na het maken van een longlist, is de volgende stap in de methodologie het maken van een shortlist op basis van de longlist. Uit deze lijst zullen drie integratie tools of maatwerkoplossingen geselecteerd worden op basis van de vereisten uit de requirements-analyse. Deze drie tools zullen elk individueel verder onderzocht worden en uitgebreid besproken. Hierbij wordt er gekeken naar wat de functionaliteit is van elke tool, waarvoor de tool hoofdzakelijk wordt gebruikt, hoe de software in elkaar zit en het welke kosten er aan iedere tool verbonden zijn.

\section{Proof of concept}
\label{sec:Proof of conceptBP}

\subsection{Uitleg}
\label{sec:UitlegBP}

Voor de proof of concept van dit onderzoek zullen de integratie tools uit de shortlist getest worden om te zien welke tool het best voldoet aan de vooropgestelde eisen uit de requirements analyse. In de shortlist werd dit gedaan op een theoretische wijze, maar in de proof of concept worden deze tools praktisch getest door ze te gebruiken om een hypothetisch bedrijfsproces te optimaliseren. De exacte vereisten en welk proces zal worden getest met deze tools zal bepaald worden in samenspraak met Axians zodat het proof of concept toepasselijk is op de huidige situatie van het bedrijf.

\subsection{Manier van onderzoek}
\label{sec:Manier van onderzoekBP}

Bij de start van de proof of concept wordt er eerst samengezeten met de stakeholders van \\Axians om samen een praktische opdracht uit te werken die toepasselijk is om de integratie tools op een correcte manier te testen. Deze gesprekken en de feedbackloop voor het uitwerken van het plan van aanpak zal 7 dagen in beslag nemen. Na het opstellen van een praktische opdracht zullen de integratie tools individueel onderzocht worden om te testen hoe deze exact werken zodat er daarna vlot aan de proof of concept gewerkt kan worden. Tijdens deze praktische uitwerking wordt de exacte tijdsduur van de opdracht gelogd, terwijl iedere integratietool wordt geëvalueerd op basis van de criteria uit de requirements-analyse en persoonlijke ervaringen met iedere tool.

\section{Verwerking proof of concept data}
\label{sec:Verwerking proof of concept dataBP}

Na een theoretisch en praktisch onderzoek te doen naar de integratie tools worden de eindresultaten geëvalueerd. Hiervoor worden de eindproducten van iedere tool met elkaar vergeleken en krijgen ze elk een score op basis van vooropgestelde criteria uit de requirements-analyse en de persoonlijke feedback van de stakeholders van Axians. Indien de omzetting van een bedrijfsproces aan de hand van een integratie tool niet volledig is afgewerkt, wordt dit ook opgenomen in het rapport.

\section{Conclusie proof of concept}
\label{sec:Conclusie proof of conceptBP}

Na het verwerken van de data en het evalueren van de integratie tools wordt in de laatste fase van de proof of concept conclusies getrokken zodat de sterktes en gebreken van iedere tool duidelijk worden blootgelegd. Vervolgens worden deze tools gerangschikt op basis van de theoretische en praktische vereisten uit de requirements-analyse. Hierbij wordt ook rekening gehouden met onderzoeksvraag en alle subvragen. Deze conclusies samen met alle bevindingen worden uiteindelijk dan voorgelegd aan de stakeholders van Axians als een eindrapport.



% Voeg hier je eigen hoofdstukken toe die de ``corpus'' van je bachelorproef
% vormen. De structuur en titels hangen af van je eigen onderzoek. Je kan bv.
% elke fase in je onderzoek in een apart hoofdstuk bespreken.

%\input{...}
%\input{...}
%...

%%=============================================================================
%% Conclusie
%%=============================================================================

\chapter{Conclusie}%
\label{ch:conclusie}

% TODO: Trek een duidelijke conclusie, in de vorm van een antwoord op de
% onderzoeksvra(a)g(en). Wat was jouw bijdrage aan het onderzoeksdomein en
% hoe biedt dit meerwaarde aan het vakgebied/doelgroep? 
% Reflecteer kritisch over het resultaat. In Engelse teksten wordt deze sectie
% ``Discussion'' genoemd. Had je deze uitkomst verwacht? Zijn er zaken die nog
% niet duidelijk zijn?
% Heeft het onderzoek geleid tot nieuwe vragen die uitnodigen tot verder 
%onderzoek?

Dit onderzoek tracht een antwoord te geven op de onderzoeksvraag: “Welke integratie tool is het meest geschikt voor Axians om interne processen
te optimaliseren en te stroomlijnen tijdens de overstap naar het nieuwe ERP-
pakket?”. Om een antwoord te kunnen geven op deze onderzoeksvraag is een vergelijkende studie opgesteld die 3 integratie tools geselecteerd uit een longlist op een theoretische en een praktische manier test en vergelijkt. Voor het vergelijken van de integratie tools is een requirementsanalyse en een proof of concept opgesteld. Hierbij werd er rekening gehouden met de hoofd- en subvragen vanuit de inleiding.

\section{Antwoord op de onderzoeksvraag en alle subvragen}%
\label{Antwoord onderzoeksvraag en subvragen}

Op vlak van veiligheid en betrouwbaarheid presteren Boomi en Microsoft Power Automate zeer goed. Beide beschikken over veiligheidscertificaten, een sterke data-encryptie en zijn compliant met de Europese-GDPR wetgeving. Microsoft Power Automate blonk hierbij uit met zijn verschillende veiligheidscertificaten en encryptie die gekoppeld zijn aan het Azure platform. Zapier scoorde op dit vlak veel lager door een gebrek aan een officieel ISO-certificaat. Ook waren er bezorgdheden bij Zapier over de exacte veiligheid van confidentiële data die over het platform kan worden verstuurd.

\vspace{\baselineskip}

Op vlak van uitbreidbaarheid en schaalbaarheid presteert Boomi het beste voor middelgrote ondernemingen die nood hebben aan grootschalige en complexe data integraties, gevolgd door Microsoft PA en tenslotte Zapier. Boomi met zijn abonnementsstructuur biedt veel flexibiliteit op vlak van uitbreidbaarheid en schaalbaarheid met hun abonnementsstructuur die op maat kan worden bepaald. Boomi focust met hun data integratie platform op het aanbieden van enterprise-level data integraties. Microsoft Power Automate biedt eerder beperkte mogelijkheden op vlak van uitbreidbaarheid en schaalbaarheid. Zo is het onduidelijk hoe flexibel het platform is buiten de vooraf opgestelde abonnementsstructuren, want Microsoft Power Automate beschikt niet over een abonnement dat op maat kan opgesteld worden. Op vlak van data integratie binnen het Microsoft systeem biedt Microsoft Power Automate uitstekende prestaties, maar kunnen data integraties buiten het ecosysteem beperkingen oplopen. Zapier biedt met zijn abonnementsstructuur een goede flexibiliteit op vlak van uitbreidbaarheid en schaalbaarheid, maar bleek het platform onvoldoende geschikt om aan de eisen van een middelgrote onderneming te voldoen die Enterprise-level data integraties wil implementeren. 

\vspace{\baselineskip}

Op vlak van kosten lijkt Zapier de goedkoopste optie te zijn, gevolgd door Microsoft Power Automate en tenslotte Boomi. Ondanks dat Zapier de goedkoopste optie blijkt te zijn, is er een enorme beperking in functionaliteit die deze iets lagere prijs niet kan rechtvaardigen. Zo beschikt Zapier over zeer beperkte monitoring en logging, moet het converteren van data zo goed als zelf door de gebruiker gedaan worden en zijn de bestaande connectoren te simplistisch voor het aanmaken van Enterprise-level data integratie. Microsoft Power Automate was de middennoot op vlak van prijs met een vaste abonnementsprijs. Op vlak van functionaliteit biedt Microsoft Power Automate goede features voor het aanmaken van custom connectoren en beschikt het over tools die complexere data integraties kunnen aanmaken. Enkel op vlak van monitoring en logging kan het platform tegenvallen met zijn beperkte middelen voor diepgaande analyses. Boomi bleek uiteindelijk de duurste en vaagste optie te zijn op vlak van prijs, maar beschikt wel over de meeste functionaliteit op vlak van connecties, conversies en dat monitoring en logging. Boomi toont geen vaste prijzen voor zijn abonnementen en geeft enkel een prijs voor zijn pay-as-you-go model waarbij nog eens extra verbruikskosten worden aangerekend die niet duidelijk worden vermeld. Ondanks deze vaagheid in prijs beschikt Boomi over de beste middelen voor het aanmaken en beheren van enterprise-level data integraties.

\vspace{\baselineskip}

Met al deze observaties kunnen we concluderen dat op een algemeen niveau Boomi de beste optie is voor Axians om aan Enterprise-level data integratie te doen voor het optimaliseren en stroomlijnen van hun interne bedrijfsprocessen. Microsoft Power Automate kan ook nog dienen als een goede integratie tool, maar dan vooral voor het optimaliseren van bedrijfsprocessen tussen applicaties en systemen in het Microsoft ecosysteem. Helaas kan Zapier niet aangeraden worden als oplossing doordat de data integratie tool niet beschikt over de nodige eisen om Axians te helpen in het optimaliseren en stroomlijnen van bedrijfsprocessen.

\section{Toekomst van dit onderzoek}%
\label{Toekomst van dit onderzoek}

Voor de toekomst van dit onderzoek raad ik Axians aan om Boomi en Microsoft Power Automate verder te bestuderen om te zien hoe goed deze integratie tools kunnen voldoen aan de exacte wensen en vereisten van Axians. Zo kan het zeker interessant zijn om deze tools te testen op een grotere schaal en/of met complexere data integratie om te zien of deze integratie tools nuttig kunnen zijn voor Axians. Ook kan het interessant zijn om deze integratie tools verder te vergelijken met andere data integratie platformen die niet geselecteerd werden uit de longlist, zoals Mulesoft of APPSeCONNECT om te zien of deze integratie tools in aanmerking komen voor gebruikt te worden.

\newpage

\begin{landscape}

\section{Gecombineerd eindoverzicht van alle integratie tools}%
\label{EindoverzichtTools}
    
\begin{table}[H]
\centering
\resizebox{\linewidth}{!}{% resize the table to fit page width
\begin{tabular}{|llll|}
\hline
\multicolumn{1}{|l|}{\textbf{Requirements uit de requirementsanalyse met scores tussen 1 en 5}}                                                                                                                            & \multicolumn{1}{l|}{\textbf{Boomi}} & \multicolumn{1}{l|}{\textbf{Zapier}} & \textbf{MicrosoftPA} \\ \hline
\textbf{Must have}                                                                                                                                                                                                         &                                     &                                      &                       \\ \hline
\multicolumn{1}{|l|}{Veiligheid en betrouwbaarheid: De tool moet voldoen aan de relevante beveiligingsnormen (Welbekende ISO-normen, data-encryptie, Europese GDPR-wetgeving compliance).}                                 & \multicolumn{1}{l|}{4}              & \multicolumn{1}{l|}{2}               & 5                     \\ \hline
\multicolumn{1}{|l|}{Uitbreidbaarheid: De tool moet eenvoudig uitbreidbaar zijn met nieuwe modules en integraties naarmate de organisatie groeit.}                                                                         & \multicolumn{1}{l|}{4}              & \multicolumn{1}{l|}{3}               & 4                     \\ \hline
\multicolumn{1}{|l|}{Integratie van meerdere systemen: Het platform moet in staat zijn om verschillende applicaties en systemen met elkaar te verbinden (zoals ERP, CRM, HRM, legacy-systemen, databases) met elkaar te verbinden.} & \multicolumn{1}{l|}{4}              & \multicolumn{1}{l|}{2}               & 3                     \\ \hline
\multicolumn{1}{|l|}{Automatisering van processen: Manuele administratieve taken moeten geautomatiseerd kunnen worden om fouten te verminderen en bepaalde processen te versnellen.}                                       & \multicolumn{1}{l|}{4}              & \multicolumn{1}{l|}{3}               & 4                     \\ \hline
\multicolumn{1}{|l|}{Data-transformatie: Het platform moet data makkelijk kunnen converteren tussen verschillende formaten en structuren (bijv. XML ↔ JSON, CSV ↔ database records).}                                      & \multicolumn{1}{l|}{5}              & \multicolumn{1}{l|}{1}               & 3                     \\ \hline
\multicolumn{1}{|l|}{Monitoring en logging: Het systeem moet real-time monitoring en logboek functionaliteiten bieden om fouten en prestaties bij te houden.}                                                              & \multicolumn{1}{l|}{5}              & \multicolumn{1}{l|}{3}               & 3                     \\ \hline
\textbf{Should have}                                                                                                                                                                                                       &                                     &                                      &                       \\ \hline
\multicolumn{1}{|l|}{Kostenbeheersing: De tool moet betaalbaar zijn met een transparant kostenmodel (licenties, onderhoud, implementatie).}                                                                                & \multicolumn{1}{l|}{3}              & \multicolumn{1}{l|}{4}               & 4                     \\ \hline
\multicolumn{1}{|l|}{Schaalbaarheid: De tool moet eenvoudig schaalbaar zijn naargelang de wensen van de organisatie.}                                                                                                      & \multicolumn{1}{l|}{4}              & \multicolumn{1}{l|}{3}               & 3                     \\ \hline
\multicolumn{1}{|l|}{Connectoren en adapters: Het platform moet kant-en-klare connectoren bieden voor veelgebruikte systemen (bijv. SAP, Microsoft Dynamics, Oracle).}                                                     & \multicolumn{1}{l|}{4}              & \multicolumn{1}{l|}{2}               & 4                     \\ \hline
\multicolumn{1}{|l|}{Prestaties: De tool moet performant zijn bij het uitvoeren van bepaalde veelgebruikte data integraties, maar niet alle data integratie processen.}                                                    & \multicolumn{1}{l|}{4}              & \multicolumn{1}{l|}{3}               & 3                     \\ \hline
\multicolumn{1}{|l|}{Onderhoudsvriendelijkheid: De tool moet eenvoudig te updaten en te onderhouden zijn met minimale downtime.}                                                                                           & \multicolumn{1}{l|}{4}              & \multicolumn{1}{l|}{3}               & 3                     \\ \hline
\textbf{Could have}                                                                                                                                                                                                        &                                     &                                      &                       \\ \hline
\multicolumn{1}{|l|}{Hybrid en multi-cloud ondersteuning: Het platform moet applicaties kunnen integreren die draaien in verschillende omgevingen (on-premise, private cloud, public cloud).}                              & \multicolumn{1}{l|}{4}              & \multicolumn{1}{l|}{2}               & 4                     \\ \hline
\multicolumn{1}{|l|}{Low-code / no-code interface: Het platform beschikt over een low-code of no-code interface voor het aanmaken van data integraties.}                                                                   & \multicolumn{1}{l|}{4}              & \multicolumn{1}{l|}{3}               & 4                     \\ \hline
\multicolumn{1}{|l|}{\textbf{Gemiddelde}}                                                                                                                                                                                  & \multicolumn{1}{l|}{4,08}              & \multicolumn{1}{l|}{2,62}               & 3,62                     \\ \hline
\multicolumn{1}{|l|}{\textbf{Mediaan}}                                                                                                                                                                                     & \multicolumn{1}{l|}{4}              & \multicolumn{1}{l|}{3}               & 4                     \\ \hline
\end{tabular}
}
\caption{Eindoverzicht van alle integratie tools}
\end{table}

\end{landscape}

%---------- Bijlagen -----------------------------------------------------------

\appendix

\chapter{Onderzoeksvoorstel}

Het onderwerp van deze bachelorproef is gebaseerd op een onderzoeksvoorstel dat vooraf werd beoordeeld door de promotor. Dat voorstel is opgenomen in deze bijlage.

%% TODO: 
%\section*{Samenvatting}

% Kopieer en plak hier de samenvatting (abstract) van je onderzoeksvoorstel.

% Verwijzing naar het bestand met de inhoud van het onderzoeksvoorstel
%---------- Inleiding ---------------------------------------------------------

% TODO: Is dit voorstel gebaseerd op een paper van Research Methods die je
% vorig jaar hebt ingediend? Heb je daarbij eventueel samengewerkt met een
% andere student?
% Zo ja, haal dan de tekst hieronder uit commentaar en pas aan.

%\paragraph{Opmerking}

% Dit voorstel is gebaseerd op het onderzoeksvoorstel dat werd geschreven in het
% kader van het vak Research Methods dat ik (vorig/dit) academiejaar heb
% uitgewerkt (met medesturent VOORNAAM NAAM als mede-auteur).
% 

\section{Inleiding}%
\label{sec:inleiding}

\subsection{Probleemstelling}
\label{sec:Probleemstelling}

Axians maakt deel uit van de Vinci energies groep en is een IT provider met zowel software als hardware implementaties en stapt begin 2025 over naar een nieuw ERP pakket. Omwille van interne groei, overnames en organisatorische wijzigingen staan er een aantal manuele en niet-manuele administratieve processen onder druk. Dit zorgt voor fouten en onnodige stress bij medewerkers. Volgens Axians zijn niet alle bedrijfsprocessen en gebruikte tools momenteel goed op elkaar afgestemd. Hierdoor kwam er een verzoek van het bedrijf om onderzoek te doen naar verschillende integraties tools op de markt en deze te vergelijken met elkaar. Bij dit onderzoek moet er hoofdzakelijk nadruk gelegd worden op de integratiemogelijkheden, functionaliteit, bruikbaarheid, onderhoudsvriendelijkheid, veiligheid en recurrente kosten. Dit omvat ook een overweging van een maatwerkoplossing voor integraties die specifiek aansluit op de behoeften van Axians.

\subsection{Onderzoeksvraag}
\label{sec:Onderzoeksvraag}

Gezien integratie tools een belangrijk aspect vormen van het ERP-proces, is het interessant om te onderzoeken. Welke integratie tool het meest geschikt is om Axians te helpen bij het optimaliseren van hun bedrijfsprocessen. De focus ligt hierbij op het vinden van een integratie tool die voldoet aan alle eisen van een grote internationale onderneming. Hiervoor luidt volgende onderzoeksvraag:

\begin{itemize}
  \item Welke integratie tool of maatwerkoplossing is het meest geschikt voor Axians om interne processen te optimaliseren en te stroomlijnen tijdens de overstap naar het nieuwe ERP-pakket?
\end{itemize}

\subsubsection{Subvragen}
\label{sec:Subvragen}
\begin{itemize}
  \item Hoe goed zorgen de tools voor veiligheid en betrouwbaarheid bij de uitwisseling van data tussen systemen?
  \item Welke integratie tool biedt de meeste mogelijkheden voor uitbreidbaarheid?
  \item Wat zijn de lange termijn kosten van elke tool, vergeleken met een maatwerkoplossing?
  \item Welke integratie tool biedt Axians de meeste functionaliteit?
\end{itemize}
 
\subsection{Onderzoeksdoelstelling}
\label{sec:Onderzoeksdoel}

Dit onderzoek oogt op het ondersteunen van Axians bij het vinden van de optimale integratie tool voor het optimaliseren van het ERP-proces en ook algemene bedrijfsprocessen te verbeteren. Dit zal op zijn beurt dan ook ervoor moeten zorgen dat het bedrijf beter zal kunnen omgaan met interne groei en de last ervan op de administratie.

\subsection{Structuur van het onderzoek}
\label{sec:Structuur van het onderzoek}

\begin{itemize}
  \item \hyperref[sec:literatuurstudie]{Hoofdstuk 2} bespreekt de literatuurstudie, definities en stand van zaken rondom ERP integratie tools. Dit onderdeel verduidelijkt wat ERP is en hoe integratie tools gebruikt worden om dit proces te optimaliseren en te verbeteren.
  \item \hyperref[sec:methodologie]{Hoofdstuk 3} geeft uitleg over de methodologie die gebruikt zal worden tijdens dit onderzoek. Eerst wordt de requirements-analyse besproken. Vervolgens worden de longlist en de shortlist besproken. Daarna wordt de proof of concept van dit onderzoek besproken. Tenslotte wordt er nog gesproken over de verwerking van de data die uit de proof of concept en de conclusie die daaruit voortvloeien.
  \item \hyperref[sec:Verwachte resultaten]{Hoofdstuk 4} bespreekt de verwachte resultaten die verzameld zijn op basis van de literatuurstudie en de methodologie. Dit schept een duidelijk beeld van de informatie die met dit onderzoek is verkregen.
  \item \hyperref[sec:discussie-conclusie]{Hoofdstuk 5} bespreekt de verwachte conclusies uit het onderzoek. In deze conclusie wordt een antwoord gegeven op de onderzoeksvraag en de bijbehorende subvragen.
  
\end{itemize}
%---------- Stand van zaken ---------------------------------------------------

\section{Literatuurstudie}%
\label{sec:literatuurstudie}

In het eerste hoofdstuk van de literatuurstudie worden enkele belangrijke termen omtrent ERP en integratie tools gedefinieerd. Vervolgens wordt het belang van ERP in grote ondernemingen besproken samen met een korte geschiedenis van hoe bedrijven vroeger werkten voor het ontstaan van ERP en hoe andere systemen geëvolueerd zijn naar het ERP-systeem. Tenslotte wordt er ook kort besproken welke verbeteringen er nog kunnen worden toegepast op ERP om het proces nog te optimaliseren en te verbeteren aan de hand van nieuwe technologieën.

\subsection{Definities}
\label{sec:Definities}

\textbf{ERP}: Een ERP, ofwel enterprise resource planning verwijst naar de software die bedrijven gebruiken voor het uitvoeren van hun dagdagelijkse administratieve bedrijfsactiviteiten, zoals in de boekhouding of bij de aankoop of verkoop van goederen. Wat ook vaak gekoppeld wordt aan ERP is enterprise performance management, software die helpt bij het plannen, budgetteren, voorspellen en rapporteren van de financiële resultaten van een organisatie. \autocite{Oracle2017}

\vspace{\baselineskip}

ERP-systemen zorgen ervoor dat een grote hoeveelheid aan bedrijfsprocessen worden samengebracht en gesynchroniseerd met elkaar. Door de gedeelde transactiegegevens van een organisatie uit meerdere bronnen te verzamelen, elimineren ERP-systemen dubbele gegevens en bieden ze gegevensintegriteit met één enkele bron van waarheid. \autocite{Oracle2017}

\vspace{\baselineskip}

Volgens \textcite{Oracle2017}: zijn ERP-systemen essentieel bij grote of internationale bedrijven voor de correcte verwerking van alle administratieve taken binnen deze ondernemingen. \textcite{Oracle2017} beweert zelf dat ERP even onmisbaar is als elektriciteit voor bedrijven van deze schaal.

\vspace{\baselineskip}

\textbf{Integratie tool}: Een integratie tool is een instrument of software dat gebruikt wordt voor het combineren van gegevens uit verschillende ongelijksoortige bronnen om gebruikers een overzichtelijk beeld te geven van deze informatie. Integratie tools brengen kleinere componenten in één systeem samen zodat deze als één geheel kunnen samenwerken. \autocite{Microsoft2024}

\vspace{\baselineskip}

Volgens \textcite{Microsoft2024} helpen integratie tools bij het consolideren van alle soorten gegevens, gezien de groei, het volume en de verschillende formaten. \textcite{Microsoft2024} beweert ook door deze te combineren om te werken met één set gegevens, dat bedrijven interne afdelingen kunnen helpen om oog in oog te staan met strategieën en zakelijke beslissingen en bruikbare en overtuigende zakelijke inzichten te produceren voor succes op korte en lange termijn.



\subsection{Het belang van ERP-systemen in een ondernemening}
\label{sec:Het belang van ERP-systemen in een ondernemening}

ERP-systemen combineren het concept van bedrijfsproces integratie met een technisch platform dat bestaat uit een geïntegreerde database en modules voor verschillende functionele domeinen. ERP-systemen hebben hun oorsprong in de vroege jaren van de informatica in de jaren 1940 en zijn geëvolueerd via geïntegreerde controletools (1960s) en Material Requirements Planning (MRP)-systemen (1970s en 1980s). In de jaren 1990 tot 2000 kenden ERP-systemen aanvankelijk een monolithische architectuur, die vanaf de 2010s plaatsmaakte voor postmoderne ERP-systemen met meerdere platformen. \autocite{katuu2020enterprise}

\vspace{\baselineskip}

Deze evolutie weerspiegelt de aanpassingen van ERP-systemen aan interne en externe uitdagingen binnen ondernemingen, zoals stijgende verwachtingen van stakeholders en klanten, terwijl beschikbare middelen afnemen. Voor effectieve integratie en waardecreatie moeten ERP-systemen ingebed worden in een technologisch ecosysteem dat rekening houdt met institutionele strategieën en operaties. Hierbij is het essentieel om over te stappen van traditionele monolithische systemen naar cloudgebaseerde en postmoderne ERP-platformen die compatibel zijn met innovaties zoals kunstmatige intelligentie en Robotic Process Automation. \autocite{katuu2020enterprise}

\vspace{\baselineskip}

Deze inzichten onderstrepen de noodzaak voor organisaties om hun ERP-systemen voortdurend te innoveren en af te stemmen op nieuwe technologische mogelijkheden.

\vspace{\baselineskip}

ERP speelt ook een cruciale rol bij het integreren van informatie en processen binnen en tussen de verschillende afdelingen van een onderneming. Dit is met name waardevol voor grote organisaties met complexe structuren en uiteenlopende operaties. Oorspronkelijk waren ERP-systemen gericht op het ondersteunen van interne operationele processen, maar hun functionaliteit is aanzienlijk uitgebreid. Tegenwoordig functioneren ze als platforms die de gehele bedrijfsvoering kunnen verbinden en integreren met andere bedrijfstoepassingen, zoals Supply Chain Management (SCM) en Customer Relationship Management (CRM). \autocite{sheik2020enterprise}

\vspace{\baselineskip}

De brede toepasbaarheid van ERP-systemen heeft hun implementatie over diverse industrieën gestimuleerd en geleid tot toenemende aandacht van management experts en onderzoekers. Hoewel er al veel vooruitgang is geboekt, biedt het onderwerp nog steeds talrijke mogelijkheden voor verder onderzoek vanuit verschillende invalshoeken. Verdere studies kunnen bijdragen aan een beter begrip en innovatieve toepassingen van ERP binnen uiteenlopende organisatorische contexten. \autocite{sheik2020enterprise}

\subsection{Mogelijke verbeteringen in het ERP-proces}
\label{sec:Mogelijke verbeteringen in het ERP-proces}

Het bewustzijn van organisaties over veranderingen in ERP-systemen speelt een cruciale rol in het verhogen van klanttevredenheid. Slimme apparaten die realtime gegevens verschaffen over producten, kwaliteit en transport hebben niet alleen een aanzienlijke impact op de klantenservice, maar verbeteren ook het algemene organisatie management. Vooral de integratie van cloud-ERP met IoT biedt veelbelovende mogelijkheden voor zowel efficiënter management als verbeterde klantgerichte dienstverlening. \autocite{tavana2020iot}

% Voor literatuurverwijzingen zijn er twee belangrijke commando's:
% \autocite{KEY} => (Auteur, jaartal) Gebruik dit als de naam van de auteur
%   geen onderdeel is van de zin.
% \textcite{KEY} => Auteur (jaartal)  Gebruik dit als de auteursnaam wel een
%   functie heeft in de zin (bv. ``Uit onderzoek door Doll & Hill (1954) bleek
%   ...'')

%---------- Methodologie ------------------------------------------------------
\section{Methodologie}%
\label{sec:methodologie}

\subsection{Requirements analyse}
\label{sec:Requirements analyse}

De eerste stap in het effectieve onderzoek is het opstellen van een requirement-analyse, waarbij de eisen voor de integratie tools worden opgelijst en gerangschikt volgens de MoSCoW-methode. Voor het opstellen van een concrete en volledige lijst worden twee bronnen geraadpleegd. Ten eerste wordt voorgaande literatuurstudie gebruikt om bepaalde noodzakelijke eisen te identificeren. Deze eerste stap kan 2 dagen duren. Daarna worden gesprekken gevoerd met de interne stakeholders van Axians, zodat eventuele aanvullingen, nieuwe eisen en exacte parameters worden toegevoegd. Voor deze gesprekken wordt 1 dag gerekend. Tenslotte wordt er nog één dag voorzien voor het finaliseren van de requirements-analyse.

\subsection{Longlist}
\label{sec:Longlist}

Na het samenstellen van een requirementsanalyse is het belangrijk een longlist van integratie tools en eventuele maatwerkoplossingen samen te stellen op basis van deze
analyse en de informatie uit de literatuurstudie. Voor het onderzoeken van deze tools en maatwerkoplossingen en het samenstellen van de longlist is een tijdsduur van 5 dagen voorzien. Deze lijst wordt vervolgens gerangschikt aan de hand van de volledige lijst van requirements uit de requirements-analyse. Deze stap zal één dag in beslag nemen, afhankelijk van de grootte van de lijst requirements. De rangschikking wordt weergegeven in een tabel, waarbij de best scorende tools en maatwerkoplossingen zich bovenaan bevinden.

\subsection{Shortlist}
\label{sec:Shortlist}

Vervolgens, na het maken van een longlist, is de volgende stap in de methodologie het maken van een shortlist op basis van de longlist. Uit deze lijst zullen drie integratie tools of maatwerkoplossingen geselecteerd worden op basis van de vereisten uit de requirements-analyse. Deze drie tools zullen elk individueel verder onderzocht worden en uitgebreid besproken. Hierbij wordt er gekeken naar wat de functionaliteit is van elke tool, waarvoor de tool hoofdzakelijk wordt gebruikt, hoe de software in elkaar zit en het welke kosten er aan iedere tool verbonden zijn. Deze shortlist zal 4 dagen in beslag nemen.

\subsection{Proof of concept}
\label{sec:Proof of concept}

\subsubsection{Uitleg}
\label{sec:Uitleg}

Voor de proof of concept van dit onderzoek zullen de integratie tools uit de shortlist getest worden om te zien welke tool het best voldoet aan de vooropgestelde eisen uit de requirements analyse. In de shortlist werd dit gedaan op een theoretische wijze, maar in de proof of concept worden deze tools praktisch getest door ze te gebruiken om een hypothetisch bedrijfsproces te optimaliseren. De exacte vereisten en welk proces zal worden getest met deze tools zal bepaald worden in samenspraak met Axians zodat het proof of concept toepasselijk is op de huidige situatie van het bedrijf.

\subsubsection{Manier van onderzoek}
\label{sec:Manier van onderzoek}

Bij de start van de proof of concept wordt er eerst samengezeten met de stakeholders van \\Axians om samen een praktische opdracht uit te werken die toepasselijk is om de integratie tools op een correcte manier te testen. Deze gesprekken en de feedbackloop voor het uitwerken van het plan van aanpak zal 7 dagen in beslag nemen. Na het opstellen van een praktische opdracht zullen de integratie tools individueel onderzocht worden om te testen hoe deze exact werken zodat er daarna vlot aan de proof of concept gewerkt kan worden. Dit zal voor iedere tool 2 dagen in beslag nemen. Tenslotte wordt er 7 dagen gerekend om ieder platform uit te testen aan de hand van de proof of concept. Tijdens deze praktische uitwerking wordt de exacte tijdsduur van de opdracht gelogd, terwijl iedere integratietool wordt geëvalueerd op basis van de criteria uit de requirements-analyse en persoonlijke ervaringen met iedere tool.

\subsection{Verwerking proof of concept data}
\label{sec:Verwerking proof of concept data}

Na een theoretisch en praktisch onderzoek te doen naar de integratie tools worden de eindresultaten geëvalueerd. Hiervoor worden de eindproducten van iedere tool met elkaar vergeleken en krijgen ze elk een score op basis van vooropgestelde criteria uit de requirements-analyse en de persoonlijke feedback van de stakeholders van Axians. Indien de omzetting van een bedrijfsproces aan de hand van een integratie tool niet volledig is afgewerkt, wordt dit ook opgenomen in het rapport. Deze verwerking zal 3 dagen in beslag nemen.

\subsection{Conclusie proof of concept}
\label{sec:Conclusie proof of concept}

Na het verwerken van de data en het evalueren van de integratie tools wordt in de laatste fase van de proof of concept conclusies getrokken zodat de sterktes en gebreken van iedere tool duidelijk worden blootgelegd. Vervolgens worden deze tools gerangschikt op basis van de theoretische en praktische vereisten uit de requirements-analyse. Hierbij wordt ook rekening gehouden met onderzoeksvraag en alle subvragen. Deze conclusies samen met alle bevindingen worden uiteindelijk dan voorgelegd aan de stakeholders van Axians als een eindrapport. Deze conclusie zal 3 dagen in beslag nemen.

%---------- Verwachte resultaten ----------------------------------------------
\section{Verwachte resultaten}
\label{sec:Verwachte resultaten}

Het verwachte resultaat is een rapport en een proof of concept dat per criteria één bepaalde integratie tool naar boven schuift als meest geschikte voor dit aspect. Deze integratie tools zullen vervolgens verder besproken worden om te verklaren waarom zij het beste zijn voor hun respectievelijke categorie. Aan de hand van deze resultaten zal dan ook een antwoord kunnen worden gegeven op de onderzoeksvraag en de subvragen. Axians zal uiteindelijk dan zelf uitmaken welk integratie tool het beste past voor hun ERP-proces. Voor iedere integratie tool zal ook een ranking gegeven worden van hoe goed ze gepresteerd hebben in de verschillende categorieën om te bepalen welke tool het meest geschikt is om te voldoen aan de eisen die opgesteld zijn in dit onderzoek.

\section{Discussie, verwachte conclusie}
\label{sec:discussie-conclusie}

De opgeleverde resultaten kunnen Axians ondersteunen in hun zoektocht naar de ideale integratie tool voor het optimaliseren van het ERP-proces. Verdere interne onderzoeken zullen op basis van dit onderzoek moeten uitmaken welke integratie tool daadwerkelijk de beste optie blijkt voor het bedrijf. Dit zal er uiteindelijk voor zorgen dat de administratieve last verlaagd wordt en dat het bedrijf beter zal kunnen omgaan met interne groei.



%%---------- Andere bijlagen --------------------------------------------------
% TODO: Voeg hier eventuele andere bijlagen toe. Bv. als je deze BP voor de
% tweede keer indient, een overzicht van de verbeteringen t.o.v. het origineel.
%\input{...}

%%---------- Backmatter, referentielijst ---------------------------------------

\backmatter{}

\setlength\bibitemsep{2pt} %% Add Some space between the bibliograpy entries
\printbibliography[heading=bibintoc]

\end{document}
