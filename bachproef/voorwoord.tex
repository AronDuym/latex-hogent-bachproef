%%=============================================================================
%% Voorwoord
%%=============================================================================

\chapter*{\IfLanguageName{dutch}{Woord vooraf}{Preface}}%
\label{ch:voorwoord}

%% TODO:
%% Het voorwoord is het enige deel van de bachelorproef waar je vanuit je
%% eigen standpunt (``ik-vorm'') mag schrijven. Je kan hier bv. motiveren
%% waarom jij het onderwerp wil bespreken.
%% Vergeet ook niet te bedanken wie je geholpen/gesteund/... heeft

Voor u ligt mijn bachelorproef met als titel ‘Een onderzoek en evaluatie van integratie tools voor ERP optimalisatie en de stroomlijning van bedrijfsprocessen in een snelgroeiende IT-omgeving’. Deze bachelorproef is de laatste stap in het afsluiten van mijn opleiding ‘Bachelor Toegepaste Informatica’ aan de Hogeschool Gent.

\vspace{\baselineskip}

Het schrijven van deze bachelorproef bleek een zeer uitdagende maar ook zeer leerrijke opdracht te zijn. Ik heb tijdens het proces en het opstellen van dit document veel geleerd over de wereld van integratie tools in de wereld van IT2Business en over mezelf als student die dit domein moest onderzoeken.

\vspace{\baselineskip}

Als eerste wil ik Laurens Coppens bedanken voor deze bachelorproef mogelijk te maken. Het is dankzij zijn connectie dat ik in contact ben gekomen met mijn co-promotor en dit onderzoek tot stand is gekomen.

\vspace{\baselineskip}

Vervolgens wil ik mijn co-promotor Frederik Trenson bedanken voor de deskundige begeleiding, waardevolle steun en feedback gedurende het opstellen van deze bachelorproef alsook voor zijn hulp bij het uitwerken van de praktische proef.

\vspace{\baselineskip}

Daarnaast wil ik ook mijn promotor Martine Van Audenrode bedanken voor haar waardevolle feedback en begeleiding gedurende het schrijven van deze bachelorproef.

\vspace{\baselineskip}

Ten slotte wil ik ook mijn mede-studenten, lectoren en ouders bedanken voor hun steun en aanmoediging tijdens dit hele proces.

\vspace{\baselineskip}

Als laatste hoop ik dat deze bachelorproef Axians een beetje kan helpen in hun zoektocht naar de ideale oplossing om hun administratieve bedrijfsprocessen te kunnen verbeteren aan de hand van een integratie tool.

\vspace{\baselineskip}

Aron Duym
