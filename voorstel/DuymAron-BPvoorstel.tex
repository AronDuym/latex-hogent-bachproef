%==============================================================================
% Sjabloon onderzoeksvoorstel bachproef
%==============================================================================
% Gebaseerd op document class `hogent-article'
% zie <https://github.com/HoGentTIN/latex-hogent-article>

% Voor een voorstel in het Engels: voeg de documentclass-optie [english] toe.
% Let op: kan enkel na toestemming van de bachelorproefcoördinator!
\documentclass{hogent-article}

% Invoegen bibliografiebestand
\addbibresource{voorstel.bib}

% Informatie over de opleiding, het vak en soort opdracht
\studyprogramme{Professionele bachelor toegepaste informatica}
\course{Bachelorproef}
\assignmenttype{Onderzoeksvoorstel}
% Voor een voorstel in het Engels, haal de volgende 3 regels uit commentaar
% \studyprogramme{Bachelor of applied information technology}
% \course{Bachelor thesis}
% \assignmenttype{Research proposal}

\academicyear{2024-2025} % TODO: pas het academiejaar aan

% TODO: Werktitel
\title{Een onderzoek en evaluatie van integratie tools voor ERP optimalisatie en de stroomlijning van bedrijfsprocessen in een snelgroeiende IT-omgeving}

% TODO: Studentnaam en emailadres invullen
\author{Aron Duym}
\email{aron.duym@student.hogent.be}


% TODO: Geef de co-promotor op
\supervisor[Co-promotor]{L. Coppens (Axians, \href{mailto:laurens.coppens@axians.com}{laurens.coppens@axians.com})}

% Binnen welke specialisatierichting uit 3TI situeert dit onderzoek zich?
% Kies uit deze lijst:
%
% - Mobile \& Enterprise development
% - AI \& Data Engineering
% - Functional \& Business Analysis
% - System \& Network Administrator
% - Mainframe Expert
% - Als het onderzoek niet past binnen een van deze domeinen specifieer je deze
%   zelf
%
\specialisation{Functional \& Business Analysis}
\keywords{Enterprise resource planning, Integratie tools, Procesoptimalisatie}

\begin{document}

\begin{abstract}
  Axians, onderdeel van de Vinci Energies Group, is een IT-provider die begin 2025 de overstap zal maken naar een nieuw ERP-systeem. Deze transitie legt druk op bestaande administratieve processen, zowel handmatige als geautomatiseerde, die momenteel niet optimaal zijn afgestemd op de groei en de interne structuur van het bedrijf. Dit onderzoek richt zich op de vraag welke integratie tool of maatwerkoplossing Axians het beste kan ondersteunen bij het stroomlijnen en optimaliseren van interne processen in samenhang met het nieuwe ERP-systeem. Deze integratie tools worden vervolgens getest in een proof of concept waarbij de functionaliteiten en prestaties van iedere tool geëvalueerd worden op basis van een hypothetisch bedrijfsproces. De verwachte resultaten tonen een rangschikking van de integratie tools op basis van criteria zoals veiligheid, functionaliteit, en kosten, met een specifieke aanbeveling per categorie. Deze evaluatie zal Axians voorzien van een duidelijke keuze voor een integratie-oplossing die de interne processen optimaliseert en de administratieve belasting vermindert. De conclusie van dit onderzoek stelt Axians in staat om de best passende integratie tool te selecteren, waardoor de organisatie beter kan inspelen op interne groei en veranderingen, met als einddoel de administratieve efficiëntie te verhogen en fouten te verminderen.
\end{abstract}

\tableofcontents

% De hoofdtekst van het voorstel zit in een apart bestand, zodat het makkelijk
% kan opgenomen worden in de bijlagen van de bachelorproef zelf.
%---------- Inleiding ---------------------------------------------------------

% TODO: Is dit voorstel gebaseerd op een paper van Research Methods die je
% vorig jaar hebt ingediend? Heb je daarbij eventueel samengewerkt met een
% andere student?
% Zo ja, haal dan de tekst hieronder uit commentaar en pas aan.

%\paragraph{Opmerking}

% Dit voorstel is gebaseerd op het onderzoeksvoorstel dat werd geschreven in het
% kader van het vak Research Methods dat ik (vorig/dit) academiejaar heb
% uitgewerkt (met medesturent VOORNAAM NAAM als mede-auteur).
% 

\section{Inleiding}%
\label{sec:inleiding}

\subsection{Probleemstelling}
\label{sec:Probleemstelling}

Axians maakt deel uit van de Vinci energies groep en is een IT provider met zowel software als hardware implementaties en stapt begin 2025 over naar een nieuw ERP pakket. Omwille van interne groei, overnames en organisatorische wijzigingen staan er een aantal manuele en niet-manuele administratieve processen onder druk. Dit zorgt voor fouten en onnodige stress bij medewerkers. Volgens Axians zijn niet alle bedrijfsprocessen en gebruikte tools momenteel goed op elkaar afgestemd. Hierdoor kwam er een verzoek van het bedrijf om onderzoek te doen naar verschillende integraties tools op de markt en deze te vergelijken met elkaar. Bij dit onderzoek moet er hoofdzakelijk nadruk gelegd worden op de integratiemogelijkheden, functionaliteit, bruikbaarheid, onderhoudsvriendelijkheid, veiligheid en recurrente kosten. Dit omvat ook een overweging van een maatwerkoplossing voor integraties die specifiek aansluit op de behoeften van Axians.

\subsection{Onderzoeksvraag}
\label{sec:Onderzoeksvraag}

Gezien integratie tools een belangrijk aspect vormen van het ERP-proces, is het interessant om te onderzoeken. Welke integratie tool het meest geschikt is om Axians te helpen bij het optimaliseren van hun bedrijfsprocessen. De focus ligt hierbij op het vinden van een integratie tool die voldoet aan alle eisen van een grote internationale onderneming. Hiervoor luidt volgende onderzoeksvraag:

\begin{itemize}
  \item Welke integratie tool of maatwerkoplossing is het meest geschikt voor Axians om interne processen te optimaliseren en te stroomlijnen tijdens de overstap naar het nieuwe ERP-pakket?
\end{itemize}

\subsubsection{Subvragen}
\label{sec:Subvragen}
\begin{itemize}
  \item Hoe goed zorgen de tools voor veiligheid en betrouwbaarheid bij de uitwisseling van data tussen systemen?
  \item Welke integratie tool biedt de meeste mogelijkheden voor uitbreidbaarheid?
  \item Wat zijn de lange termijn kosten van elke tool, vergeleken met een maatwerkoplossing?
  \item Welke integratie tool biedt Axians de meeste functionaliteit?
\end{itemize}
 
\subsection{Onderzoeksdoelstelling}
\label{sec:Onderzoeksdoel}

Dit onderzoek oogt op het ondersteunen van Axians bij het vinden van de optimale integratie tool voor het optimaliseren van het ERP-proces en ook algemene bedrijfsprocessen te verbeteren. Dit zal op zijn beurt dan ook ervoor moeten zorgen dat het bedrijf beter zal kunnen omgaan met interne groei en de last ervan op de administratie.

\subsection{Structuur van het onderzoek}
\label{sec:Structuur van het onderzoek}

\begin{itemize}
  \item \hyperref[sec:literatuurstudie]{Hoofdstuk 2} bespreekt de literatuurstudie, definities en stand van zaken rondom ERP integratie tools. Dit onderdeel verduidelijkt wat ERP is en hoe integratie tools gebruikt worden om dit proces te optimaliseren en te verbeteren.
  \item \hyperref[sec:methodologie]{Hoofdstuk 3} geeft uitleg over de methodologie die gebruikt zal worden tijdens dit onderzoek. Eerst wordt de requirements-analyse besproken. Vervolgens worden de longlist en de shortlist besproken. Daarna wordt de proof of concept van dit onderzoek besproken. Tenslotte wordt er nog gesproken over de verwerking van de data die uit de proof of concept en de conclusie die daaruit voortvloeien.
  \item \hyperref[sec:Verwachte resultaten]{Hoofdstuk 4} bespreekt de verwachte resultaten die verzameld zijn op basis van de literatuurstudie en de methodologie. Dit schept een duidelijk beeld van de informatie die met dit onderzoek is verkregen.
  \item \hyperref[sec:discussie-conclusie]{Hoofdstuk 5} bespreekt de verwachte conclusies uit het onderzoek. In deze conclusie wordt een antwoord gegeven op de onderzoeksvraag en de bijbehorende subvragen.
  
\end{itemize}
%---------- Stand van zaken ---------------------------------------------------

\section{Literatuurstudie}%
\label{sec:literatuurstudie}

In het eerste hoofdstuk van de literatuurstudie worden enkele belangrijke termen omtrent ERP en integratie tools gedefinieerd. Vervolgens wordt het belang van ERP in grote ondernemingen besproken samen met een korte geschiedenis van hoe bedrijven vroeger werkten voor het ontstaan van ERP en hoe andere systemen geëvolueerd zijn naar het ERP-systeem. Tenslotte wordt er ook kort besproken welke verbeteringen er nog kunnen worden toegepast op ERP om het proces nog te optimaliseren en te verbeteren aan de hand van nieuwe technologieën.

\subsection{Definities}
\label{sec:Definities}

\textbf{ERP}: Een ERP, ofwel enterprise resource planning verwijst naar de software die bedrijven gebruiken voor het uitvoeren van hun dagdagelijkse administratieve bedrijfsactiviteiten, zoals in de boekhouding of bij de aankoop of verkoop van goederen. Wat ook vaak gekoppeld wordt aan ERP is enterprise performance management, software die helpt bij het plannen, budgetteren, voorspellen en rapporteren van de financiële resultaten van een organisatie. \autocite{Oracle2017}

\vspace{\baselineskip}

ERP-systemen zorgen ervoor dat een grote hoeveelheid aan bedrijfsprocessen worden samengebracht en gesynchroniseerd met elkaar. Door de gedeelde transactiegegevens van een organisatie uit meerdere bronnen te verzamelen, elimineren ERP-systemen dubbele gegevens en bieden ze gegevensintegriteit met één enkele bron van waarheid. \autocite{Oracle2017}

\vspace{\baselineskip}

Volgens \textcite{Oracle2017}: zijn ERP-systemen essentieel bij grote of internationale bedrijven voor de correcte verwerking van alle administratieve taken binnen deze ondernemingen. \textcite{Oracle2017} beweert zelf dat ERP even onmisbaar is als elektriciteit voor bedrijven van deze schaal.

\vspace{\baselineskip}

\textbf{Integratie tool}: Een integratie tool is een instrument of software dat gebruikt wordt voor het combineren van gegevens uit verschillende ongelijksoortige bronnen om gebruikers een overzichtelijk beeld te geven van deze informatie. Integratie tools brengen kleinere componenten in één systeem samen zodat deze als één geheel kunnen samenwerken. \autocite{Microsoft2024}

\vspace{\baselineskip}

Volgens \textcite{Microsoft2024} helpen integratie tools bij het consolideren van alle soorten gegevens, gezien de groei, het volume en de verschillende formaten. \textcite{Microsoft2024} beweert ook door deze te combineren om te werken met één set gegevens, dat bedrijven interne afdelingen kunnen helpen om oog in oog te staan met strategieën en zakelijke beslissingen en bruikbare en overtuigende zakelijke inzichten te produceren voor succes op korte en lange termijn.



\subsection{Het belang van ERP-systemen in een ondernemening}
\label{sec:Het belang van ERP-systemen in een ondernemening}

ERP-systemen combineren het concept van bedrijfsproces integratie met een technisch platform dat bestaat uit een geïntegreerde database en modules voor verschillende functionele domeinen. ERP-systemen hebben hun oorsprong in de vroege jaren van de informatica in de jaren 1940 en zijn geëvolueerd via geïntegreerde controletools (1960s) en Material Requirements Planning (MRP)-systemen (1970s en 1980s). In de jaren 1990 tot 2000 kenden ERP-systemen aanvankelijk een monolithische architectuur, die vanaf de 2010s plaatsmaakte voor postmoderne ERP-systemen met meerdere platformen. \autocite{katuu2020enterprise}

\vspace{\baselineskip}

Deze evolutie weerspiegelt de aanpassingen van ERP-systemen aan interne en externe uitdagingen binnen ondernemingen, zoals stijgende verwachtingen van stakeholders en klanten, terwijl beschikbare middelen afnemen. Voor effectieve integratie en waardecreatie moeten ERP-systemen ingebed worden in een technologisch ecosysteem dat rekening houdt met institutionele strategieën en operaties. Hierbij is het essentieel om over te stappen van traditionele monolithische systemen naar cloudgebaseerde en postmoderne ERP-platformen die compatibel zijn met innovaties zoals kunstmatige intelligentie en Robotic Process Automation. \autocite{katuu2020enterprise}

\vspace{\baselineskip}

Deze inzichten onderstrepen de noodzaak voor organisaties om hun ERP-systemen voortdurend te innoveren en af te stemmen op nieuwe technologische mogelijkheden.

\vspace{\baselineskip}

ERP speelt ook een cruciale rol bij het integreren van informatie en processen binnen en tussen de verschillende afdelingen van een onderneming. Dit is met name waardevol voor grote organisaties met complexe structuren en uiteenlopende operaties. Oorspronkelijk waren ERP-systemen gericht op het ondersteunen van interne operationele processen, maar hun functionaliteit is aanzienlijk uitgebreid. Tegenwoordig functioneren ze als platforms die de gehele bedrijfsvoering kunnen verbinden en integreren met andere bedrijfstoepassingen, zoals Supply Chain Management (SCM) en Customer Relationship Management (CRM). \autocite{sheik2020enterprise}

\vspace{\baselineskip}

De brede toepasbaarheid van ERP-systemen heeft hun implementatie over diverse industrieën gestimuleerd en geleid tot toenemende aandacht van management experts en onderzoekers. Hoewel er al veel vooruitgang is geboekt, biedt het onderwerp nog steeds talrijke mogelijkheden voor verder onderzoek vanuit verschillende invalshoeken. Verdere studies kunnen bijdragen aan een beter begrip en innovatieve toepassingen van ERP binnen uiteenlopende organisatorische contexten. \autocite{sheik2020enterprise}

\subsection{Mogelijke verbeteringen in het ERP-proces}
\label{sec:Mogelijke verbeteringen in het ERP-proces}

Het bewustzijn van organisaties over veranderingen in ERP-systemen speelt een cruciale rol in het verhogen van klanttevredenheid. Slimme apparaten die realtime gegevens verschaffen over producten, kwaliteit en transport hebben niet alleen een aanzienlijke impact op de klantenservice, maar verbeteren ook het algemene organisatie management. Vooral de integratie van cloud-ERP met IoT biedt veelbelovende mogelijkheden voor zowel efficiënter management als verbeterde klantgerichte dienstverlening. \autocite{tavana2020iot}

% Voor literatuurverwijzingen zijn er twee belangrijke commando's:
% \autocite{KEY} => (Auteur, jaartal) Gebruik dit als de naam van de auteur
%   geen onderdeel is van de zin.
% \textcite{KEY} => Auteur (jaartal)  Gebruik dit als de auteursnaam wel een
%   functie heeft in de zin (bv. ``Uit onderzoek door Doll & Hill (1954) bleek
%   ...'')

%---------- Methodologie ------------------------------------------------------
\section{Methodologie}%
\label{sec:methodologie}

\subsection{Requirements analyse}
\label{sec:Requirements analyse}

De eerste stap in het effectieve onderzoek is het opstellen van een requirement-analyse, waarbij de eisen voor de integratie tools worden opgelijst en gerangschikt volgens de MoSCoW-methode. Voor het opstellen van een concrete en volledige lijst worden twee bronnen geraadpleegd. Ten eerste wordt voorgaande literatuurstudie gebruikt om bepaalde noodzakelijke eisen te identificeren. Deze eerste stap kan 2 dagen duren. Daarna worden gesprekken gevoerd met de interne stakeholders van Axians, zodat eventuele aanvullingen, nieuwe eisen en exacte parameters worden toegevoegd. Voor deze gesprekken wordt 1 dag gerekend. Tenslotte wordt er nog één dag voorzien voor het finaliseren van de requirements-analyse.

\subsection{Longlist}
\label{sec:Longlist}

Na het samenstellen van een requirementsanalyse is het belangrijk een longlist van integratie tools en eventuele maatwerkoplossingen samen te stellen op basis van deze
analyse en de informatie uit de literatuurstudie. Voor het onderzoeken van deze tools en maatwerkoplossingen en het samenstellen van de longlist is een tijdsduur van 5 dagen voorzien. Deze lijst wordt vervolgens gerangschikt aan de hand van de volledige lijst van requirements uit de requirements-analyse. Deze stap zal één dag in beslag nemen, afhankelijk van de grootte van de lijst requirements. De rangschikking wordt weergegeven in een tabel, waarbij de best scorende tools en maatwerkoplossingen zich bovenaan bevinden.

\subsection{Shortlist}
\label{sec:Shortlist}

Vervolgens, na het maken van een longlist, is de volgende stap in de methodologie het maken van een shortlist op basis van de longlist. Uit deze lijst zullen drie integratie tools of maatwerkoplossingen geselecteerd worden op basis van de vereisten uit de requirements-analyse. Deze drie tools zullen elk individueel verder onderzocht worden en uitgebreid besproken. Hierbij wordt er gekeken naar wat de functionaliteit is van elke tool, waarvoor de tool hoofdzakelijk wordt gebruikt, hoe de software in elkaar zit en het welke kosten er aan iedere tool verbonden zijn. Deze shortlist zal 4 dagen in beslag nemen.

\subsection{Proof of concept}
\label{sec:Proof of concept}

\subsubsection{Uitleg}
\label{sec:Uitleg}

Voor de proof of concept van dit onderzoek zullen de integratie tools uit de shortlist getest worden om te zien welke tool het best voldoet aan de vooropgestelde eisen uit de requirements analyse. In de shortlist werd dit gedaan op een theoretische wijze, maar in de proof of concept worden deze tools praktisch getest door ze te gebruiken om een hypothetisch bedrijfsproces te optimaliseren. De exacte vereisten en welk proces zal worden getest met deze tools zal bepaald worden in samenspraak met Axians zodat het proof of concept toepasselijk is op de huidige situatie van het bedrijf.

\subsubsection{Manier van onderzoek}
\label{sec:Manier van onderzoek}

Bij de start van de proof of concept wordt er eerst samengezeten met de stakeholders van \\Axians om samen een praktische opdracht uit te werken die toepasselijk is om de integratie tools op een correcte manier te testen. Deze gesprekken en de feedbackloop voor het uitwerken van het plan van aanpak zal 7 dagen in beslag nemen. Na het opstellen van een praktische opdracht zullen de integratie tools individueel onderzocht worden om te testen hoe deze exact werken zodat er daarna vlot aan de proof of concept gewerkt kan worden. Dit zal voor iedere tool 2 dagen in beslag nemen. Tenslotte wordt er 7 dagen gerekend om ieder platform uit te testen aan de hand van de proof of concept. Tijdens deze praktische uitwerking wordt de exacte tijdsduur van de opdracht gelogd, terwijl iedere integratietool wordt geëvalueerd op basis van de criteria uit de requirements-analyse en persoonlijke ervaringen met iedere tool.

\subsection{Verwerking proof of concept data}
\label{sec:Verwerking proof of concept data}

Na een theoretisch en praktisch onderzoek te doen naar de integratie tools worden de eindresultaten geëvalueerd. Hiervoor worden de eindproducten van iedere tool met elkaar vergeleken en krijgen ze elk een score op basis van vooropgestelde criteria uit de requirements-analyse en de persoonlijke feedback van de stakeholders van Axians. Indien de omzetting van een bedrijfsproces aan de hand van een integratie tool niet volledig is afgewerkt, wordt dit ook opgenomen in het rapport. Deze verwerking zal 3 dagen in beslag nemen.

\subsection{Conclusie proof of concept}
\label{sec:Conclusie proof of concept}

Na het verwerken van de data en het evalueren van de integratie tools wordt in de laatste fase van de proof of concept conclusies getrokken zodat de sterktes en gebreken van iedere tool duidelijk worden blootgelegd. Vervolgens worden deze tools gerangschikt op basis van de theoretische en praktische vereisten uit de requirements-analyse. Hierbij wordt ook rekening gehouden met onderzoeksvraag en alle subvragen. Deze conclusies samen met alle bevindingen worden uiteindelijk dan voorgelegd aan de stakeholders van Axians als een eindrapport. Deze conclusie zal 3 dagen in beslag nemen.

%---------- Verwachte resultaten ----------------------------------------------
\section{Verwachte resultaten}
\label{sec:Verwachte resultaten}

Het verwachte resultaat is een rapport en een proof of concept dat per criteria één bepaalde integratie tool naar boven schuift als meest geschikte voor dit aspect. Deze integratie tools zullen vervolgens verder besproken worden om te verklaren waarom zij het beste zijn voor hun respectievelijke categorie. Aan de hand van deze resultaten zal dan ook een antwoord kunnen worden gegeven op de onderzoeksvraag en de subvragen. Axians zal uiteindelijk dan zelf uitmaken welk integratie tool het beste past voor hun ERP-proces. Voor iedere integratie tool zal ook een ranking gegeven worden van hoe goed ze gepresteerd hebben in de verschillende categorieën om te bepalen welke tool het meest geschikt is om te voldoen aan de eisen die opgesteld zijn in dit onderzoek.

\section{Discussie, verwachte conclusie}
\label{sec:discussie-conclusie}

De opgeleverde resultaten kunnen Axians ondersteunen in hun zoektocht naar de ideale integratie tool voor het optimaliseren van het ERP-proces. Verdere interne onderzoeken zullen op basis van dit onderzoek moeten uitmaken welke integratie tool daadwerkelijk de beste optie blijkt voor het bedrijf. Dit zal er uiteindelijk voor zorgen dat de administratieve last verlaagd wordt en dat het bedrijf beter zal kunnen omgaan met interne groei.



\printbibliography[heading=bibintoc]

\end{document}